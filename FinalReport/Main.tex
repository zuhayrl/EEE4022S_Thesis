% Zuhayr Loonat - University of Cape Town - LNTZUH001
% This is a project report templace document created for EEE4022FS students at the University of Cape Town.
% This file should be is processed with ``pdflatex`` and might need a few modifications if a different processor is chosen.
% Use pdflatex - bibtex - pdf latex - pdflatex

\documentclass[a4paper,12pt]{report}

%Include packages you need to use here

\usepackage[top = 1in, bottom = 1in, left = 1in, right = 1in]{geometry}
\usepackage{graphicx}
\usepackage{fancyhdr}
\usepackage{amsmath, amsthm, amssymb}
\usepackage{lastpage}
\usepackage{subfigure}
\usepackage{lscape}
\usepackage{hyphenat}
\usepackage{setspace}
\usepackage{hyperref}
\usepackage{siunitx}
\usepackage{array}
\usepackage{float}

% Glossary package
% \usepackage[acronyms]{glossaries}
% \setacronymstyle{long-short}
\usepackage[acronym]{glossaries-extra}
\setabbreviationstyle[acronym]{long-short}
% \renewcommand{\acronymname}{Glossary}
\makenoidxglossaries

% Include glossary definitions
% Glossary definitions for the thesis
% This file contains all glossary entries used throughout the document

% Physical quantities and units
\newglossaryentry{voltage}{name={V}, description={Voltage (V)}}
\newglossaryentry{current}{name={I}, description={Current (A)}}
\newglossaryentry{resistance}{name={R}, description={Resistance ($\Omega$)}}
\newglossaryentry{capacitance}{name={C}, description={Capacitance (F)}}
\newglossaryentry{inductance}{name={L}, description={Inductance (H)}}
\newglossaryentry{power}{name={P}, description={Power (W)}}
\newglossaryentry{energy}{name={E}, description={Energy (J)}}
\newglossaryentry{time}{name={t}, description={Time (s)}}
\newglossaryentry{frequency}{name={f}, description={Frequency (Hz)}}
\newglossaryentry{resistivity}{name={$\rho$}, description={Resistivity ($\Omega \cdot$m) or Density (kg/m$^3$)}}
\newglossaryentry{conductivity}{name={$\sigma$}, description={Conductivity (S/m)}}
\newglossaryentry{temperature}{name={T}, description={Temperature (K or $^\circ$C)}}
\newglossaryentry{pressure}{name={p}, description={Pressure (Pa)}}
\newglossaryentry{salinity}{name={S}, description={Salinity (ppt or PSU)}}
\newglossaryentry{charge}{name={Q}, description={Charge (C)}}
\newglossaryentry{gravity}{name={g}, description={Acceleration due to gravity (9.81 m/s$^2$)}}
\newglossaryentry{viscosity}{name={$\mu$}, description={Viscosity (Pa$\cdot$s) or Permeability (H/m)}}
\newglossaryentry{psu}{name={PSU}, description={Practical Salinity Unit}}
\newglossaryentry{ppt}{name={ppt}, description={Parts per thousand}}

% Constants
\newglossaryentry{boltzmann}{name={k}, description={Boltzmann constant ($1.38 \times 10^{-23}$ J/K)}}
\newglossaryentry{faraday}{name={F}, description={Faraday constant ($96485$ C/mol)}}

% Chemical quantities
\newglossaryentry{molarity}{name={M}, description={Molarity (mol/L)}}
\newglossaryentry{moles}{name={n}, description={Number of moles (mol)}}

% Acronyms
\newacronym{eis}{EIS}{Electrical Impedance Spectroscopy}
\newacronym{ml}{ML}{Machine Learning}

% Include page formatting here. 
\parskip=6mm
\parindent=0mm
\renewcommand{\headrulewidth}{0pt}
\rhead[]{\thesection}
\lhead[\thechapter]{}


\begin{document}

% This section formats the title page of the Report.
\thispagestyle{empty}
    {\Huge \begin{center}
        % Modify the line below to insert your title.
        The Measurement of Water Salinity \\for Antarctic Research
        %\hrule 
        % Modify the line below to insert your subtitle.
        %{\Large Insert a subtitle here (if applicable)}
    \end{center}}

    \vskip 5mm
    \begin{center}
        \includegraphics[scale = 0.35]{Figures/uctLogo.png}
    \end{center}

    \vskip 5mm
    \begin{center}
        Presented by:\\
        Zuhayr Loonat		% Insert your name here
    \end{center}

    \vskip 10mm
    \begin{center}
        Prepared for:\\
        Justin Pead\\ 		% Insert your supervisor's name here.
        Dept.~of Electrical Engineering\\University of Cape Town
    \end{center}


    \vskip 10mm
    \begin{center}
        Submitted to the Department of Electrical Engineering at the University of Cape Town in partial
        fulfilment of the academic requirements for a Bachelor of Science degree in Electrical and Computer Engineering
    \end{center}


    \vskip 5mm
    \begin{center}
        {\bf \today}
    \end{center}

%second Page
%\newpage
%\thispagestyle{empty}
%\mbox{}

% Declarartion
\fancyfoot[C]{\thepage}
\pagestyle{plain}
\newpage
\pagenumbering{roman}
%\newpage
    \onehalfspacing{
    \nohyphens{
    %\thispagestyle{empty}
    \vskip 40mm


    % Please leave the declaration as it is (Standard UCT declaration).
    {\Huge{Declaration}}\\
    %\hrule

    \vskip 10mm
    \begin{enumerate}
        \item I know that plagiarism is wrong.
        Plagiarism is to use another's work and pretend that it is one's own.
        \item I have used the IEEE convention for citation and referencing.
        Each contribution to, and quotation in, this report `The Measurement of Water Salinity for Antarctic Research' from the work{(s)} of other people has been attributed and has been cited and referenced.
        Any section taken from an internet source has been referenced to that source.
        \item This report `The Measurement of Water Salinity for Antarctic Research' is my own work and is in my own words (except where I have attributed it to others).
        \item I have not paid a third party to complete my work on my behalf.
        My use of artificial intelligence software has been limited to giving overviews of topics, checking grammar, aiding with {\LaTeX} commands, and aiding with making BibTeX citations of sources (specify precisely how you used AI to assist with this assignment, and then give examples of the prompts you used in your first appendix).
        \item I have not allowed and will not allow anyone to copy my work with the intention of passing it off as his or her own work.
        \item I acknowledge that copying someone else's assignment or essay, or part of it, is wrong, and declare that this is my own work
    \end{enumerate}
    \vskip30mm
        \begin{tabular*}{\textwidth}{@{}l@{\extracolsep{\fill}}r@{}}
        Signature:\ldots\ldots\ldots\ldots\ldots\ldots\ldots\ldots\ldots & Date: {\today} \\
        Zuhayr Loonat & \\
        Word Count: & \\
        \end{tabular*}
        \vskip10mm

%\newpage
    %\thispagestyle{empty}
    %{\Huge Terms of Reference}\\
    %\hrule
    %\vskip 10mm
    %The terms of reference page is an agreement between yourself and your supervisor outlining what is expected of you in your final year project. Please make sure that this is discussed and written at the beginning of your thesis project. 


%\fancyfoot[C]{\thepage}
%\pagestyle{plain}
%\pagenumbering{roman}
\newpage
{\Huge Acknowledgments}\\
%\hrule
\vskip 10mm
I would like to express my sincere gratitude to my supervisor, Justin Pead, for his invaluable guidance, support, and expertise throughout this research.
His insights and encouragement were instrumental in bringing this work to fruition.

I am also deeply grateful to my friends for their unwavering moral support during this journey.
Their encouragement and understanding made the challenges of this process far more manageable.

Lastly, I would like to thank my family for their support during my studies.

\newpage
    {\Huge Abstract}\\
    %\hrule
    \vskip 10mm

    % Place your abstract here.
    Ice shelves form when glaciers flow from the Antarctic out over the ocean.
    This ice, which is originally freshwater ice is formed from accumulated snow, extends over the water and floats, because ice is less dense than water.
    When saltwater freezes at the bottom of the ice shelves, salt is expelled, forming a brine like solution, with a high salt density, directly under the ice shelves.

    The salinity of this brine solution and the seawater underneath need to be measured to aid in Antarctic research.
    This project documents the design and testing of a conductivity-based salinity measuring device to measure the salt content in these areas.
    The prototype was designed in two parts, a probe module, which would be lowered down through a water tower, which was drilled into the ice shelves, to measure the salinity, and a controller, which would be used to send instructions and receive data from the probe.
    The probe utilised gold electrodes to measure and analyse the salinity through conductivity.
    Two methods were investigated,~\gls{dc}, where point voltage measurements and voltage sweeps were conducted, and~\gls{ac} where a machine learning model was used to map properties of the input and output waves to a salinity value.
    [Talk about accuracy TODO]

%\newpage
    %{\Huge Glossary}\\
    %\hrule
    %\vskip 10mm
    
    %\glsaddall

\newpage
    \tableofcontents

%\newpage
%\listoffigures

%\newpage
%\listoftables

% Page formatting, to place section titles as headers of odd pages and Chapter titles as headers of even pages.
\newpage
%\fancyhead[RE,LO]{}
%\fancyhead[LE]{\leftmark}
%\fancyhead[RO]{\rightmark}
\pagestyle{fancy}

\pagenumbering{arabic}

% Print Glossary/Acronyms
\renewcommand{\acronymname}{Glossary}
\printnoidxglossaries

% THe files included below are .tex files containing the respective chapters these are already created in this package and you can add to or modify them.
\chapter{Introduction}

\section{Background to the study}
Antarctic ice shelves form where glaciers from the continent extend out over the ocean.
This creates massive floating platforms that serve as crucial boundaries between land, ice and the marine environment.
At the base of these ice shelves, the relatively warmer ocean water melts the ice above, releasing freshwater that rises upward because it is less dense than the surrounding salty seawater.
As this freshwater moves upwards and experiences less pressure, the pressure difference can allow it to become chilled below its normal freezing point without turning to ice.
When this water eventually refreezes onto the underside of the ice shelf, it creates what is known as marine ice.
Marine ice is a layer of ice that originated from seawater, whereas regular ice forms from snow that fell on land.
During this refreezing process, salt is pushed out of the water, forming ice crystals in a phenomenon known as brine rejection, which creates pockets of extremely salty, dense water that sink downwards.
This phenomenon helps drive ocean currents that continually bring new water into contact with the ice shelf base, establishing a continuous cycle of melting and freezing.
Understanding how salinity varies within and beneath ice shelves is increasingly important because changes in ocean temperature and salt content can significantly affect how quickly the ice melts from below.
This melting process can weaken ice shelves and, in extreme cases, lead to their collapse.
This, in turn, allows glaciers on land to flow more rapidly into the ocean, contributing to sea level rise.
Traditionally, scientists have measured salinity in these environments by collecting water samples at various depths.
However, this approach only provides salinity data at specific locations and times and does not allow for continuous measurement.

\section{Objectives of this study}
This study aims to design a prototype device that would allow researchers to take continuous real time salinity measurements in these `water towers'.
It aims to create a device that can accurately measure salinity, and that can be iterated upon, allowing it to be used in the harsh conditions of the Antarctic.
Additionally, this study aims to create a machine learning model, which would allow for the prediction of salinity from the electro-chemical make-up of the water, through a process called Electrochemical Impedance Spectroscopy.

\section{Scope and Limitations}
The scope of this project includes the design, assembly and testing of the salinity measuring device, as well as a review on the feasibility of using machine learning to predict salinity.
This includes researching relevant literature, to get a good understanding of similar devices that already exist and how salinity can be calulated using these devices, followed by the design process and assembly of the prototype device, and the design of the machine learning model.
This then moves to the testing of the devce and the machine learning model, to determine their effectiveness in measuring and predicting salinity, respectively.

This project must be completed in a specified time of 13 weeks, from its inception to submission.
A budget of R2000 has been imposed on the entire project. This includes design, assembly and testing.
This budget can only be spent through the Electrical Engineering Department of the University of Cape Town. 


\section{Plan of development}
This projects first starts with an literature review in Chapter 2, where salinity, its measurement methods, what \gls{eis} is and how it can be paired with \gls{ml}, are reviewed.
Chapter 3 covers the chosen methodology, covering the choice of salinity measurement method, its design and assembly.
Chapter 4 details the testing and evaluation of the chosen design, including its salinity measuring accuracy.
Chpater 5 concludes the report with a summary of the objectives and results, and Chapter 6 includes reccomendations for further studies on this work.

\chapter{Literature Review}
\section{Introduction}
Accurate salinity measurement is fundamental to oceanographic research.
Traditional measurement techniques have evolved from labour-intensive chemical titration methods to modern electronic sensors, with electrical conductivity emerging as the predominant approach due to its combination of accuracy, speed, and practical deployability.
This literature review examines the current state of salinity measurement technology with particular emphasis on conductivity-based methods and emerging machine learning approaches for electrochemical data interpretation. The review is organised into three main sections.
First, we establish the fundamental concepts of salinity and provide a comprehensive comparison of available measurement techniques.
Second, we examine the theoretical foundations and practical implementation of electrical conductivity measurements for salinity determination, including instrumentation, calibration procedures, and current limitations.
Finally, we explore the application of \gls{eis} and \gls{ml} as advanced approaches for enhanced salinity analysis, examining how frequency-domain measurements and intelligent data processing can overcome limitations of traditional single-frequency conductivity methods.
This comprehensive review provides the theoretical and methodological foundation for developing a machine learning-enhanced impedance spectroscopy approach for salinity determination.

\section{Salinity: Definition}
Salinity is a fundamental characteristic of water, and is most commonly defined as the total amount of dissolved salts in water, and in the context of oceanography, seawater.
It is typically expressed in \gls{ppt} or \gls{psu}~\cite{salinity_def}. 
The concept of salinity has evoloved significantly from its early definition, which was based on chlorinity measurements.
Modern salinty is defined through the Practical Salinity Scale 1978 (PSS-78), where salinity is based on the conductivity ratio of standard seawater solutions, to a standard Potassium Chloride solution, and is dimensionless~\cite{unesco_salinity}.
The salinity-conductivity relationship is however, quite complex, requiring corrections and calibrations needed for depth and temperature, as these both play a factor in the conductivity of the water. 

\section{Overview of Salinity Measurement Methods}
There are a multitude of methods which can be used to measure salinity, each with their own advatages, limitations and levels of accuracy.
Traditional methods include gravimetric analysis, chemical titration (such as the Mohr-Knudsen method for chlorinity), and refractometry. While these techniques can provide accurate results, they are often time-consuming, require skilled operators, and are not easily adaptable to in-situ or automated measurements.
Modern approaches predominantly rely on electrical conductivity sensors, which offer rapid, repeatable, and automated salinity determination.
Other techniques, such as optical methods and ion-selective electrodes, have also been explored, but are less commonly used in oceanographic applications due to issues with robustness, calibration, or specificity.
The choice of method depends on the required accuracy, operational environment, and available resources.


\subsection{Historical Methods}
\subsubsection{Chlorinity Titration}
Early salinity measurements relied on chemical titration methods, in particular the Mohr-Knudsen chlorinity titration, which used silver nitrate.
The chlorinity of a solution has the definition `the mass of silver required to precipitate completely the halogens in $0.3 285 234 kg$ of the ocean-water sample'
This method was highly accurate, with results within ($\pm0.001$ PSU). However, it relied heavily on toxic chemicals, and was a time-consuming laboratory procedure, with limited practical application in the field.

\subsubsection{Gravimetric Methods}
Gravimetric analysis, a technique used to determine an amount of a substance, by measuring its change in mass, invloves evaporation and the weighing of dissovled solids.
This method directly provided measurements of the salt content, within accuracies of ($\pm0.001$ PSU), under controlled laboratory conditions.
This method remains the reference standard for calibration processes, but is however, extremely slow.

\subsection{Physical Property Based Methods}
There are several methods that utilise the relationship between salinity and the physical properties of water.

\subsubsection{Hydrometric and Density Methods}
Hydrometric methods using density measurements via hydrometers, offer salinity measurements that are low-cost, and electronics free. 
However, they are limited in precision with accuracies of $\pm 1-2$ PSU, and require large sample volumes.
The hydrometer is a floating instrument, that sinks to different depths depending on the density of the solution, and by measuring how high or low it floats, the density of the solution can be determined.
The following equation is used to map the relationship between salinity and density.

\begin{equation}
    \rho = \rho_0(1+kS)
\end{equation}

where $\rho$ is the density, $\rho_0$ is the density of fresh water, $S$ is the salinity and $k$ is the proportionality constant. 

This can then be inverted to give Salinity from Density:

\begin{equation}
S = \frac{\frac{\rho}{\rho_0}-1}{k}
\end{equation}

This however, does not include temperature correction.

\subsubsection{Refractometric Thechniques}
Refractometric techniques measure the refractive index changes cuased by the dissovled salts.
The refractive index of seawater is influenced by wavelength, temperature, salinity, and pressure. 
Within the range of 500-700~nm wavelength, 0-30$^\circ$C temperature, 0-40~PSU salinity, and 0-11000~dbar pressure, the refractive index equation provides an accuracy of 0.4-80~ppm~PSU, with accuracy decreasing as pressure increases.
Refractometers, which require only a small sample volume, are compact devices, making them suitable for portable field measurements.
Fibre optic refractometers have improved portability and reduced temperature sensitivity, with moderate accuracy (±0.5-1 PSU), making them increasingly popular in aquaculture applications. 


\subsubsection{Freezing Point Osmometry}
Freezing point depression osmometry exploits the colligative (i.e.~relating to the binding together of molecules) properties of dissolved salts.
The main principle relies on freezing point depression, which is the phenomenon where a solvents freeving point is lowered when a solute is added to it.
To perform the measurement, the water is cooled till its freezing point and the temperature drop is measured, which is then used to calculate the osmolality.
This method can achieve accuracies as high as ($\pm0.001$ PSU), however its requirement for precise temperature control limits its usage to laboratory applications.

\subsubsection{Interferometry}
Interferometry is a measurement technique which measures how electro-magnetic waves are affected by changes in a solution.
Two identical light waves are passed through two solutions, one benchmark and one test sample solution.
The gain and phase shift between the waves is then used to calculate the salinity. 
This method requires precisely aligned mirrors to direct the light waves, causing it to be relatively large.

\subsubsection{Electromagnetic Induction and Magnetic Permeability}


\subsection{Advanced Analytical Methods}


\subsection{Remote Autonomous Sensing}


\section{Conductivity-Based Salinity Measurements}
\subsection{Theoretical Foundation}
Electrical conductivity has emerged as the predominant method for salinity measurement due to its practical implementation, high accuracy and fast response time.
The technique utilises the strong correlation between dissolved ionic content and electrical conductivity.

The conductivity of a liquid is measured by its ability to conduct electrical current.
The relationship between conductivity and salinity is based on the concentration of dissolved ions in seawater.
The main ions found in sea water ($Na^+, Cl^-, Mg^{2+}, {SO_4}^{2-}, Ca^{2+}, K^+$) maintain a relatively constant proportional relationship, in ocean waters~\cite{chemical_ocean}.
This enables robust corrections between conductivity and total dissolved salt content.
Unlike other measurement techniques, conductivity accounts for all the ions in the water, not only chlorine, which is why it is considered a more accurate measure of salinity~\cite{salinity_def_calc}.

The Practical Salinity Scale 1978 (PSS-78) defines Practical Salinity $S_p$ through the conductivity ratio $K_{15}$, as shown below~\cite{teos-10}:

\begin{equation}\label{eqn:k15_salinity}
    K_{15} = \frac{C(S_p, 15, 0)}{C(KCl, 15, 0)}
\end{equation}

where the numerator, $C(S, 15, 0)$ represents the conductivity of seawater sample at 15$^\circ$C and standard atmospheric pressure ($1 atm/101.325 kPa/0dbar$), and the denomimator, $C(KCl, 15, 0)$ is the conductivity of a standard $KCl$ (Potassium Chloride) solution under identical temperature and pressure.
The standard $KCl$ solution consists of $32.4356 \times 10^{-3}kg$ of $KCl$ dissolved in $1kg$ of water~\cite{lewis_pss78}.
When the ratio between the water sample and the $KCl$ solution is 1, i.e. $K_{15} = 1$, then the Practical Salinity $S_p$ is, according to the definition, 35~\cite{teos-10}.

It is important to note that Practical Salinity is a unit-less quantity, and though it may be convenient, it would be incorrect to quote it in \gls{psu}. 
Practical salinity should rather be quoted as a certain Practical Salinity `on the Practical Salinity Scale PSS-78'~\cite{teos-10}. 

When $K_{15}$ does not equal 1, Practical Salinity, $S_p$ can be calculated using the equation below~\cite{teos-10}: 
\begin{equation}\label{eqn:salinity_short}
    S_p = \sum_{i=0}^{5}a_i{(K_{15})}^{i/2} 
\end{equation}

where $K_{15}$ is the equation defined above (Equation~\ref{eqn:k15_salinity}), and the coefficients $a_i$ are given in Table (\ref{tabel:pss-78}).


\subsection{Temperature and Pressure Compensation}
When calculating salinity at conditions other than 15$^\circ$C, and $0dbar$, the conductivity ratio $R$ is expanded to the product of three ratios $R_p$, $R_t$ and $r_t$ as follows~\cite{teos-10}:
\begin{equation}
    R=\frac{C(S_p, t, p)}{C(35, 15, 0)} = R_p R_t r_t
\end{equation}

where $t$, and $p$ are the temperature and pressure valid over the ranges $-2^{\circ}C \leq t \leq 35^{\circ}C$ and $0 \leq p \leq 10 000dbar$ respectively.

These ratios can be expanded as follows:

\begin{equation}
    R = \frac{C(S_p, t, p)}{C(35, 15^{\circ} C, 0)} = \frac{C(S_p, t, p)}{C(S_p, t, 0)} \cdot \frac{C(S_p, t, 0)}{C(35, t, 0)} \cdot \frac{C(35, t, 0)}{C(35, 15^{\circ} C, 0)} = R_p R_t r_t
\end{equation}

This equation represents the ratio between the conductivity measurement of a sample $C(S_p,t,p)$ and the conductivity of the standard solution $C(35, 15^{\circ}, 0)$~\cite{teos-10}. 
In order to find the salinity, $R_p$, $R_t$ and $r_t$ need to be caculated.
First, $r_t$ is calculated using the temperature of the sample:
\begin{equation}
    r_t = \sum_{i=0}^{4} {c_i}{t_i}
\end{equation}
$R_p$ is then calculated as a function of the temperature $t$, pressure $p$, and conductivity ratio $R$:
\begin{equation}
    R_p = 1 + \frac{\sum_{i=1}^{3}{e_i}{p^i}}{1+d_1{t}+d_2{t^2}+R[d_3+d_4 {t}]}
\end{equation}
Finally, $R_t$ can be evaluated using $R$, $R_p$ and $r_t$:
\begin{equation}
    R_t = \frac{R}{{R_p}{r_t}}
\end{equation} 

At standard conditions, i.e., temperature $t=15{^\circ}$C, $R_t$ is equal to $K_{15}$ an therefore Practical salinity $S_p$ can be calculated from Equation~\ref{eqn:k15_salinity}.
For cases where the temperature is not $t=15^\circ$C, Practical Salinity $S_p$ is given as a function of $R_t$, with $k=0.0162$~\cite{teos-10}:
\begin{equation}\label{eqn:salinity_full}
S_p = \sum_{i=0}^{5} a_i {(R_t)}^{i/2} + \frac{t-15}{1+k(t-15)} \sum_{i=0}^{5} b_i {(R_t)}^{i/2}
\end{equation}

Note that Equations~(\ref{eqn:k15_salinity}) to~(\ref{eqn:salinity_full}) are only valid in the range $2 < S_p < 42$, $-2^{\circ}C \leq t \leq 35^{\circ}C$ and $0 \leq p \leq 10 000dbar$.

\begingroup
    \renewcommand{\arraystretch}{1.8} % increase row height (adjust factor as needed)
    \begin{table}[h!]
        \centering
            \begin{tabular}{|>{\centering\arraybackslash}p{1cm}|
                >{\centering\arraybackslash}m{2cm}|
                >{\centering\arraybackslash}m{2cm}|
                >{\centering\arraybackslash}m{3cm}|
                >{\centering\arraybackslash}m{3cm}|
                >{\centering\arraybackslash}m{3cm}|}
            \hline
            $i$ & $a_i$ & $b_i$ & $c_i$ & $d_i$ & $e_i$ \\ \hline
            0 & $0.0080$ & $0.0005$ & $\num{6.766097e-1}$ &  &  \\ \hline
            1 & $-0.1692$ & $-0.0056$ & $\num{2.00564e-2}$ & $\num{3.426e-2}$ & $\num{2.070e-5}$ \\ \hline
            2 & $25.3851$ & $-0.0066$ & $\num{1.104259e-4}$ & $\num{4.464e-4}$ & $\num{-6.370e-10}$ \\ \hline
            3 & $14.0941$ & $-0.0375$ & $\num{-6.9698e-7}$ & $\num{4.215e-1}$ & $\num{3.989e-15}$ \\ \hline
            4 & $-7.0261$ & $0.0636$ & $\num{1.0031e-9}$ & $\num{-3.107e-3}$ &  \\ \hline
            5 & $2.7081$ & $-0.0144$ &  &  &  \\ \hline
            \end{tabular}
        \caption{Table of Coefficients for PSS-78 Equations~\cite{teos-10}}
        \label{tabel:pss-78}
    \end{table}
\endgroup

It must be noted that the PSS-78 equations use the IPTS-68 temperature scale and in order for them to work with the current ITS-90 scale, must be converted using the equation below~\cite{teos-10}:
\begin{equation}
    t_{68}^{circ}C = 1.00024\times{t_{90}^{circ}C}
\end{equation}


\subsection{Instrumentation and Technology}
The most common method for measuring salinity is by using a~\gls{ctd} device.
The fundamental concept of these devices involves placing two electrodes in a sample of water, applying a voltage across them and measuring the water's response. This is then paired with a temperature and depth correction, allowing for an accurate salinity measurement.
The depth value for these calculations is taken from the pressure at which the measurement is taken.
This pressure is then traslated to depth using the standard depth to pressure equation.
Modern CTD systems achieve salinity accuracies better than $\pm0.005$ \gls{psu}, with some instruments like the Sea-Bird 911 Plus demonstrating historical accuracies of $\pm0.002$ \gls{psu} or $\pm0.0002$ \gls{psu}.

\subsection{Applications and Limitations}
Conductivity-based salinity measurements excel in most oceanographic and water quality applications due to its accuracy, speed, and reliability.
The conductivity method allows for real-time data capture, continuous monitoring, and easy integration with autonomous devices. 

However, this method does face some limitations. Due to its dependance on the water's capacity to conduct electricity, freshawater applications require specialised low-conductivity sensors, while hyper-saline environments could exceed the standard calibration range.

The method's reliance on emperical correlations derived from typical seawater compositions can introduce errors in waters ocean waters affected by external factors such as pollution or freshwater inflow from connecting rivers, which can alter ionic composition and introduce variability not captured by standard seawater-based calibrations.
In such environments supplementry practices may be necessary for accurate salinity measurements.
\chapter{Methodology}

\section{Salinity Measurement Method}
A \gls{ctd} sensor, which measures salinity using conductivity, temperature and depth, was chosen as the salinity measurement device.
When choosing a measurement technique multiple factors needed to be considered.
Firstly, the salinity measurements are to be conducted in the Antarctic, where the environment, and remote nature of the area, make majority of the measurement methods unusable.
Secondly, the device would need to fit through an ice core hole with a diameter of $100mm$, and lastly, the device would need to be able to take continuous measurements.

\gls{ctd} sensors do not require sample collection, unlike chlorinity titration, gravimetric analysis and refractometry.
This removes both the need for sample collection and the challenges of sample degradation, storage and transport logistics.

Modern \gls{ctd} sensors are compact, and can easily be designed for specific space constraints.
This coupled with its deployments flexibility make it the preferred choice over methods, such as laboratory methods, which suffer from deployment constraints.
\gls{ctd} sensors allow for continuous realtime monitoring, a characteristic none of the the alternative methods provide.
The alternative methods either require sample collection, or cannot measure continuously.

\gls{ctd} instruments inherently measure conductivity, temperature and pressure simultaneously, providing salinity measurements with temperature and depth compensation, whereas laboratory methods measure salinity only, and require seperate temperature measurements.


These factors coupled with the researcher's significant experience with PCB design and electronics influenced the choice for a \gls{ctd} sensor.

\section{Electrode Design}
When measuring conductivity, choosing an electrode material plays a significant role in the accuracy of the measurements.
To get an accurate measurement of the resistance of the water, ideally, a electrode resistance of zero is required.
This would allow the resistance measurement to be entirely due to the resistance of the water.
Most conductive materials have conductivities of order $10^6 - 10^8 S/m$, which is negligible compared to sea (salt) water, which has an average conductivity of $3.31 S/m$~\cite{conductivities}\cite{ocean_conductivity_tyler}.
Preferably, the material with the highest conductivity, silver, would be used. However, conductivity is not the only factor considred when designing an electrode. 
The electrodes will be submerged in saltwater, which is highly corrosive. The material chosen will require high corrosive resistance. 
Silver, though having the highest conductivity, has a low corrosion resistance, and therefore cannot be used in this application~\cite{zhang_silver}.

Titanium is the material of choice for ocean-use~\cite{materials_ocean_structures}. It is essentially corrosion-free, and offers a conductivity of $2.68\times 10^{6}$~\cite{conductivities}.
However, titanium is expensive and fell out of the budget of this project.
Gold boasts both a high conductivity of $4.10\times 10^{7}$, higher than titanium but lower than silver, and a high corrosion resistance, making it an ideal choice.
Gold is also a commonly used material in electronic design, with it being used in \gls{pcb} manufacturing, to protect copper pads from corrosion.
This is done through a \gls{enig} plating process, where a layer of nickel is chemically deposited onto the exposed copper traces, to prevent the copper from oxidizing, and then a layer of gold is applied over the nickel through an emmersion process, to protect the nickel.
This process is significantly more expensive compared to standard \gls{pcb} manufacturing, however, it allowed for the use of gold electrodes, and therefore was factored into the budget. 

In order to utilise the \gls{enig} process a \gls{pcb} was used to design the gold electrodes.
This allowed the electrodes to be designed with a known area and seperation distance, allowing for accurate conductivity calculations.
A solder pad was used to design the portion of the \gls{pcb} that would act as the electrode, since it allowed the copper/gold to be exposed.
Then during manufacturing \gls{enig} was chosen as the surface finish, to achieve the gold finish.

The \gls{pcb} was designed to allow for easy calculation of the conductivity $\sigma$, using the equation below:
\begin{equation}\label{eqn:conductivity}
    \sigma = \frac{L}{RA}
\end{equation}

where $L$ is the distance between the electrodes, $R$ is the resistance of the water, and A is the cross-sectional area of the electrodes.
A square face of $20mm\times 20mm$ was chosen to allow for easy cross-sectional area calculations, and a distance of $10mm$ was chosen as the seperation distance.
This distance was chosen as it is close enough to reduce current spreading, but not too small to where the water could not flow easily between the electrodes.
A $2mm$ fringe guard was added around the main electrode area to reduce current fringing, which is an effect that causes the current to spread beyond the edges of the gap~\cite{roshen_fringing}.
The fringe guards counteract this by saturating the area surrounding the main pads with current, preventing them from fringing.

The resistance of the electrodes was calculated using the Equation~\ref{eqn:conductivity} and was found to have an approximate resistance of $7.55\Omega$

The electrode \gls{pcb} was designed with the consideration of mounting to the probe \gls{pcb}.
To accomodate this, solder pads were added to allow the electrodes to be soldered to the probe \gls{pcb}.
Supports were also factored into the design to ensure that the electrodes stayed straight and secure.
The design can be seen in Figure [insert figure].

\section{Resistance Measurement}\label{sec:res_mes}
There are multiple ways to measure resistance, however most rely on the same principle, which is the voltage divider principle.
This principle works by using a series circuit with two resistors, and a constant known input voltage.
The voltage over each of the resistors will be proportional to their resistance, and therefore, if the resistance of one resistor is known, the resistance of the other can be calculated.
A simple voltage divider circuit can be seen in Figure~\ref{fig:voltage_divider}.
For this application the electrodes were chosen as the R2 resistor, with R1 being a large resistor of known resistance.

\begin{figure}[H]\label{fig:voltage_divider}
    \centering
    \includegraphics[width=0.6\textwidth]{figures/fig_voltage_divider.png}
    \caption{Simple voltage divider circuit used for resistance measurement.}
\end{figure}

Equations \ref{eqn:voltage_divider} and \ref{eqn:resistance_divider} are used to calculate the resistance from the voltage divider equation.
\begin{equation}\label{eqn:voltage_divider}
    V_{R2} = V_{In} \times \frac{R_2}{R_1 + R_2}
\end{equation}
\begin{equation}\label{eqn:resistance_divider}
    R_2 = \frac{R_1 \times V_{R2}}{V_{In}-V_{R2}}
\end{equation}


\section{Circuit Design}
\subsection{Probe Circuit}
The probe circuit is the circuit which contains the resistor divider, was designed to be printed onto a \gls{pcb}.
This design was influenced by Reference~\cite{cam_clark}, where a similar device was designed for salinity measurements in ice columns.
A \gls{pcb} was chosen for this circuit as the researcher had significant experience with \gls{pcb} design, and the manufacturing process offered higher precision than hand soldering, and is relatively cost-effective.
Significant improvements and modifications were made to the resistor divider circuit, to allow for a wider range of testing.

For input power, a \gls{dac} was used to drive the circuit. This allowed to the input voltage to be varied between $0 V$ and the referencevoltage, which was chosen to be $5V$.
This allowed for a range of voltages to be applied, which allowed for the measurement of the water's voltage-resistance relationship, and the creation of \gls{ac} signals.
A function generator was considered for generating the \gls{ac} signal, as it would allow for signals of a wider frequency and and high precision, however the price could not be accomodated by the budget.
The choice of \gls{dac}, and all following components, was first influenced by availability on JLCPCB, the \gls{pcb} manufacturing house.
The MCP4725 was chosen for its high resolution of 12-bits, offering a digital range of 0-4095, fast update time of $6{\mu}s$, and interface speed of 3.4MHz.
These features allow for both \gls{dc} and \gls{ac} signal analysis.

An op-amp with unity gain was the connected to the output of the \gls{dac}.
This is because \gls{dac}s have limited output drive capabilities, and the op-amp would allow for heavier loads to be driven.
Additionally the op-amp offers improved output stability, introduces impedance isolation, which protects the \gls{dac} from load variations and feedback effects, and allows for better sine wave quality.

As mentioned in Section~\ref{sec:res_mes}, for the resistor divider circuit, the electrodes would serve as R2 and a known resistor as R1.
Three alernative values of R1 were chosen, to accomodate for any circuit errors.
These could be switched between using the TS3A4751 multiplexer \gls{ic}.
This switching multiplexer was chosen, for its low on-state resistance of $0.9\Omega$, and fast switching speed of $4-5ns$~\cite{cam_clark}.

The R1 resistor values were chosen to be $100\Omega$, $1K\Omega$ and $10K\Omega$.
These values would be used when the resistance between the probes was $1-10\Omega$, $10-100\Omega$, and $100-1K\Omega$ respectively.
Each \gls{ic} contained 4 switches.

For measuring the output resistor, the voltage over it was directed into a multiplying op-amp with a gain of 11.
This increases the resolution for the \gls{adc} readings, as low voltages may be hard to differentiate between when converted to digital data. 

This configuration would allow for a minumum resolution of $11\%$ of $V_{DAC}$ and maximum of $100\%$ of $V_{DAC}$, for the voltage measurement by the \gls{adc}, as shown in Equations~\ref{eqn:dac_11} and~\ref{eqn:dac_100}~\cite{cam_clark}.
Equations~\ref{eqn:dac_11} and~\ref{eqn:dac_100} show for the expected resistance of $7.55\Omega$ falling into the $1-10\Omega$ range.
However, if the resistance falls into the $10-100\Omega$, or $100-1K\Omega$, the respective R1 resistors would be used and the maximum and minimum \gls{dac} resolutions would be the same.

\begin{equation}\label{eqn:dac_11}
    \frac{1\Omega}{1\Omega + 100\Omega}\times V_{DAC} \times 11 = 11\% V_{DAC}
\end{equation}

\begin{equation}\label{eqn:dac_100}
    \frac{10\Omega}{10\Omega + 100\Omega}\times V_{DAC} \times 11 = 100\% V_{DAC}
\end{equation}

The accuarcy of the R1 resistor is integral to acheiving an accurate R2 measurement.
The resistors available on JLCPCB had an acurracy of $\pm{1}\%$.
To increase the accuracy 3 equal resistors were put in parallel. This decreases the uncertainty of the total equivalent resistance~\cite{cam_clark}.
This is shown in Equations~\ref{eqn:parallel_r} to~\ref{eqn:parallel_uncertainty}.

\begin{equation}\label{eqn:parallel_r}
    R_{Equivalent} = {\left[{\sum_{i=1}^{n}{\frac{1}{R_n}}}\right]}^{-1}
\end{equation}
If all the Resistors are equal this simplifies to: 
\begin{equation}\label{eqn:parallel_r_equal}
    R_{Equivalent} = ({\frac{n}{R}})^{-1} = \frac{1}{n} \times R
\end{equation}
To propogate uncertainty the standard equation for combined uncertainty can be used:

If a quantity $y$ depends on several independent variables $x_1, x_2, ..., x_n$: \\
$y=f(x_1, x_2, ..., x_n)$ \\
and each $x_i$ has a standard uncertainty $u(x_i)$ then the combined standard uncertainty of $y$, denoted $u_c(y)$, is:
\begin{equation}\label{standard_uncertainty}
    u_c(y)=\root{\sum_{i=1}^{n}{({\frac{\partial{f}}{\partial{x_i}}}{u(x_i)})}^2}
\end{equation}

\chapter{Testing and Evaluation}

To properly evaluate the system, multiple testing procedures were implemented.
These started with first testing the accuracy of individual components on the \gls{pcb}, and then testing the probes ability to measure salinity, through voltage measurements relating to conductivity.
Additionally, tests were done on the \gls{ml} model programmed to map the salinity.

\section{Component and Equipment Testing}
Before the probe could be used to measure salinity, the accuracy of its components needed to be tested.
These procedures were completed before the probe was encased in resin, as access to the circuitry was required.

\subsection{Resistor Testing}
For accurate electrode resistance to be measured, the the $R1$ parallel resistor combinations would need to be measured.
As shown in Section~\ref{sec:circuit_design}, by calculation, the $R1$ resistors should have equivalent resistances of $100\Omega$, $1K\Omega$ and $10K\Omega$, with an uncertainty of $\pm0.33\%$.
These were measured using the Keysight Technologies U3401A multimeter, which had a resistance accuracy of $0.1\%$.
This multimeter would be used for all subsequent \gls{dc} voltage measurements, and has a voltage accuracy of $0.02\%$.
The multimeter probes had a resistance of $0.154\Omega$ which was accounted for.
The $R1$ resistors were measured and are shown in Table~\ref{table:resistance_test}.


\begingroup
    \renewcommand{\arraystretch}{1.8} % increase row height (adjust factor as needed)
    \begin{table}[h!]
        \centering
            \begin{tabular}{|>{\centering\arraybackslash}p{5cm}|
                >{\centering\arraybackslash}m{5cm}|}
            \hline
            Theoretical R ($\Omega$) & Measured R ($\Omega$) \\ \hline
            $99.67-100.33$ & $99.888$ \\ \hline
            $996.7-1003.3$ & $1000.146$ \\ \hline
            $9967-10033$ & $10005.746$ \\ \hline
            \end{tabular}
        \caption{Table of $R_1$ resistor measurements}
        \label{table:resistance_test}
    \end{table}
\endgroup

The calibration resistor with an expected resistance of $5\Omega\pm0.25\%$ was measured to have a resistance of $5.142\Omega$.
Taking into account the probe resistance, the calibration resistor had a resistance of $4.988\Omega$.

\subsection{DAC and ADC Accuracy}
Both the accuracy of the \gls{dac} and \gls{adc} needed to be measured as these were used for the output and measurement, respectively.

The first test was done by programming the \gls{dac} to output from its minimum to maximum value.
This would allow for the evaluation of the \gls{dac}s output offset and gain to be measured.
The \gls{adc}s were also used to measure the output of the \gls{dac}, and these measurements were compared relative to the voltage measured by the multimeter.
Figure~\ref{fig:dac_test} shows the relationship between the voltage inputted to the \gls{dac}, and its output voltage, with the output measured on the multimeter.
Note that the reference voltage of the \gls{dac} was measured to be $5.001V$.

\begin{figure}[H]
    \centering
    \includegraphics[width=0.7\textwidth]{figures/dac_test.png}
    \caption{DAC Output Voltage vs Input Voltage}
    \label{fig:dac_test}
\end{figure}

Based on the measurements made by the multimeter, the \gls{dac} had a output range of $0.0098V - 4.91V$, an offset of $0.0098V$ and gain of $0.98688$.

The accuracy of the \gls{adc} was then tested by comparing the \gls{dac} output measured by the \gls{adc} and multimeter.
Note, this \gls{adc} measurement was taken after the unity gain buffer op-amp.
For this test, the \gls{adc} took 5 measurements at each voltage step, which were taken at $1\mu{s}$ interval. 
These 5 values were averaged to give the voltage at that step.
The results of this test can be seen in Figure~\ref{fig:adc_test}.

\begin{figure}[H]
    \centering
    \includegraphics[width=0.7\textwidth]{figures/adc_test.png}
    \caption{DAC Output Measured by the ADC vs Multimeter}
    \label{fig:adc_test}
\end{figure}

Once the voltage measured by the \gls{adc} reaches $3.3V$ the \gls{adc} saturates as its reference voltage is $3.3V$.
The gain of the \gls{adc} was calculated to be 1.28072 compared to the multimeter.
The \gls{adc} had a measured gain of 1.3.

\subsection{Accuracy of Resistance Measuring Circuitry}\label{sec:resistor_measuring_test}
In order to evaluate the resistance circuit's ability to accurately measure resistance, resistors were attached to the electrode's solder pads.
This resistor acted as the $R2$ resistor and its value was calculated using Equation~\ref{eqn:resistance_divider}.
The resistance was calculated using the voltage sweep and single voltage functions mentioned in Section~\ref{sec:uc_program} with some slight adjustments for calculating resistance only.
These values were then compared to a multimeter measurement of the resistors, and the probe resistance taken into account.
Resistances were measured at $0\Omega$, or a short circuit, and then $10-82\Omega$ using resistors from the E12-Series, with an accuracy of $\pm5\%$
The $100\Omega$ $R1$ resistor was used.
The outcome of this test can be seen in Table~\ref{table:resistance_measurement_test}.

\begingroup
    \renewcommand{\arraystretch}{1.8} % increase row height (adjust factor as needed)
    \begin{table}[h!]
        \centering
            \begin{tabular}{|>{\centering\arraybackslash}p{4cm}|
                >{\centering\arraybackslash}m{5cm}|
                >{\centering\arraybackslash}m{6cm}|}
            \hline
                \textbf{Multimeter Resistance $\Omega$} & \textbf{Measured R $\Omega$} & \textbf{Acceptable Range $\Omega$} \\ \hline
                0 & 0 & 0-0 \\ \hline
                9.848 & 9.99578925 & 9.5-10.5 \\ \hline
                11.972 & 12.0007881 & 11.4-12.6 \\ \hline
                15.124 & 15.0062442 & 14.25-15.75 \\ \hline
                18.872 & 18.1212224 & 17.1-18.9 \\ \hline
                22.004 & 22.0162646 & 20.9-23.1 \\ \hline
                27.101 & 26.9989572 & 25.65-28.35 \\ \hline
                33.012 & 33.0181212 & 31.35-34.65 \\ \hline
                39.201 & 39.0305398 & 37.05-40.95 \\ \hline
                47.100 & 47.0431559 & 44.65-49.35 \\ \hline
                56.023 & 56.0306769 & 53.2-58.8 \\ \hline
                68.014 & 68.0599057 & 64.6-71.4 \\ \hline
                79.785 & 78.208607 & 77.9-86.1 \\ \hline
            \end{tabular}
        \caption{Table for Resistor Measurement Test}
        \textit{Note 1: For this test an input of 2V was used} \\
        \textit{Note 2: Acceptable range indicates resistance values due to $\pm5\%$ accuracy}
        \label{table:resistance_measurement_test}
    \end{table}
\endgroup

The voltage sweep test, from 0-2V, achieved a similar measuring accuracy as seen in Figure~\ref{fig:resistance_measurement_test}
\begin{figure}[H]
    \centering
    \includegraphics[width=0.7\textwidth]{figures/resistance_measurement_test.png}
    \caption{Resistance Measurement Test via Voltage Sweep}
    \label{fig:resistance_measurement_test}
\end{figure}

For both these tests, voltage calibration via the calibration resistor was done to ensure accurate voltage measurements.

\section{Salinity Testing}
In order for the probe to conduct salinity based tests, it was cast in epoxy as described in Section~\ref{sec:waterproofing}.
Following this a range of tests were conducted, ranging from testing voltage measurements on saline solutions, to measuring salinity via conductivity.

\subsection{Voltage Measurement Accuracy and Repeatability}
In order to get an understanding of how the electrodes interact with saline solutions, a voltage sweep test was conducted multiple times with the same solution.
This was done using the voltage sweep function, mentioned in Section~\ref{sec:uc_program}, with some alterations, allowing for the function to return the raw voltage instead of the conductivity.
The results showed that on the same solution, the the voltage sweep had the same effects.
However, it was noticed that when a voltage reading was taken in quick succession to another there was a slight interference was caused by the water, to counteract this a 1 second delay was introduced between each measurement.
After this delay was added, the interference was no longer observed.

Figure~\ref{fig:repeatability_test} shows voltage sweeps across the same solution on three separate occasions.
\begin{figure}[H]
    \centering
    \includegraphics[width=0.7\textwidth]{figures/repeatability_test.png}
    \caption{Repeatability Test Results}
    \label{fig:repeatability_test}
\end{figure}

\subsection{Conductivity and Salinity Measurement}

\subsubsection{Obtaining Conductivity of the Standard Solution}
For the measurement of salinity from conductivity, the conductivity of the standard solution of $35$ \gls{psu} at $15^{\circ}C$ and $0dbar$ must first be obtained.
To evaluate this, both the voltage sweep and single voltage measurements wer taken in a solution at standard conditions.
To achieve these conditions salt was mixed into water until the salinity was 34.8 \gls{psu}.
This value was used, as creating a solution of a specific salinity is a time-consuming process, and it was considered close enough for this experiment.
To achieve a temperature of $15^\circ$C the water was cooled in a fridge to $4^\circ$C and then left out until it reached $15^\circ$C.
The salinity was measured using a salinometer.
Once the required conditions were achieved, a voltage sweep from $0-2V$ was done using the previously mentioned voltage sweep function.
As mentioned in Section~\ref{sec:uc_program}, in the voltage sweep description, this returns the conductivity and resistance for each step.
All measurements for this and subsequent tests were conducted at $0dbar$, while using the $100\Omega$ $R_1$ resistor as the measured resistances were between $1-15\Omega$.
The sweep was conducted twice to ensure repeatability.
From this test the average conductivity of the standard solution was found to be $3.53 S/m$, with an average electrode/water resistance of $6.81\Omega$.

Similar results were obtained using single voltage measurements, where multiple readings were taken at one voltage using the DC Single Voltage function, and this was done for voltages of $1-1.5V$.
Here the average resistance was calculated to be $7.39\Omega$ and average conductivity of $3.53 S/m$.
These values correlate well with the expected resistance of $7.55\Omega$.

The Single Voltage Test can be seen in Table~\ref{table:sal_vsingledc}, with a graph illustrating the Voltage vs Resistance, taken from the Voltage Sweep Test, shown in Figure~\ref{fig:sal_vsweepdc}.

\begingroup
    \renewcommand{\arraystretch}{1.8} % increase row height (adjust factor as needed)
    \begin{table}[h!]
        \centering
            \begin{tabular}{|>{\centering\arraybackslash}p{1cm}|
                >{\centering\arraybackslash}m{1cm}|
                >{\centering\arraybackslash}m{1.8cm}|
                >{\centering\arraybackslash}m{2.2cm}|
                >{\centering\arraybackslash}m{3cm}|
                >{\centering\arraybackslash}m{3cm}|
                >{\centering\arraybackslash}m{2cm}|}
            \hline
                \textbf{V IN (V)} & \textbf{Vp AMP (V)} & \textbf{Calib F} & \textbf{Probe V (V)} & \textbf{Resistance ($\Omega$)} & \textbf{Conductivity (mS/cm)} \\ \hline
                1.2 & 1.112 & 0.7786 & 0.078709382 & 7.0195345 & 3.561489724 \\ \hline
                1.2 & 0.721 & 0.7786 & 0.051033691 & 4.4417047 & 3.628469589 \\ \hline
                1.3 & 1.355 & 0.7739 & 0.095330409 & 7.9130471 & 3.159195482 \\ \hline
                1.3 & 1.251 & 0.7739 & 0.088013536 & 7.2619240 & 3.442613812 \\ \hline
                1.4 & 1.312 & 0.7739 & 0.092305164 & 7.0586150 & 3.541770538 \\ \hline
                1.4 & 1.452 & 0.7739 & 0.102154800 & 7.8711082 & 3.176172238 \\ \hline
                1.5 & 1.573 & 0.7745 & 0.110753500 & 7.9721993 & 3.135897151 \\ \hline
                1.5 & 1.359 & 0.7745 & 0.095685955 & 6.8137148 & 3.669007482 \\ \hline
                1.6 & 2.211 & 0.7724 & 0.155525240 & 10.745988 & 3.226447904 \\ \hline 
                1.6 & 1.451 & 0.7724 & 0.101886582 & 6.8009925 & 3.675934042 \\ \hline
                \textbf{Mean} &  &  &  & \textbf{7.38991} & \textbf{3.531706371} \\ \hline
            \end{tabular}
        \caption{Table for Standard Salinity Solution Test}
        \textit{Note: For this test an R1 resistance of 100$\Omega$ was used.}
        \label{table:sal_vsingledc}
    \end{table}
\endgroup

\begin{figure}[H]
    \centering
    \includegraphics[width=0.7\textwidth]{figures/sal_vsweep.png}
    \caption{Volatge Sweep Test Showing Resistance vs Input Voltage}
    \label{fig:sal_vsweepdc}
\end{figure}

\subsubsection{Measuring Salinity of Sample Solutions}
Once the conductivity of the standard solution was found, the PSS-78 salinity equations could be used to find the salinity of sample solutions.
The both the DC Single Voltage and DC Sweep Voltage functions were updated to return the salinity of a measured solution.
For the DC Single Voltage Test, a voltage of $1.4V$ was found to return the most accurate value.
Using these methods salinity of solutions were tested against a salinometer and compared, were the single voltage test proved to give the more accurate measurements.
The comparisons for the Single Voltage test can be seen in Table~\ref{table:salinity_measurements}.

\begingroup
    \renewcommand{\arraystretch}{1.8} % increase row height (adjust factor as needed)
    \begin{table}[H]
        \centering
            \begin{tabular}{|>{\centering\arraybackslash}p{1.5cm}|
                >{\centering\arraybackslash}m{1cm}|
                >{\centering\arraybackslash}m{1.5cm}|
                >{\centering\arraybackslash}m{2cm}|
                >{\centering\arraybackslash}m{2.5cm}|
                >{\centering\arraybackslash}m{3cm}|
                >{\centering\arraybackslash}m{2cm}|}
            \hline
                \textbf{Salinity (PSU)} & \textbf{T (°C)} & \textbf{Probe Voltage} & \textbf{Calib Factor} & \textbf{Calibrated Voltage} & \textbf{Resistance ($\Omega)$} & \textbf{Calculated Salinity} \\ \hline
                34.8  & 15    & 0.119 & 0.7739 & 0.0920941 & 7.041339901  & 35    \\ \hline
                30.1  & 15    & 0.145 & 0.7745 & 0.1123025 & 8.721186459  & 28.02 \\ \hline
                23.74 & 15    & 0.188 & 0.7601 & 0.1428988 & 11.36732667  & 20.71 \\ \hline
                23.72 & 24.31 & 0.108 & 0.7687 & 0.0830196 & 6.30378402   & 25.82 \\ \hline
                32.65 & 24.27 & 0.084 & 0.7693 & 0.0646212 & 4.839166235  & 35.15 \\ \hline
                15.8  & 20    & 0.197 & 0.7772 & 0.1531084 & 12.27920695  & 14.95 \\ \hline
                20.4  & 20    & 0.163 & 0.7779 & 0.1267977 & 9.958959389  & 18.75 \\ \hline
                17.26 & 20    & 0.197 & 0.7799 & 0.1536403 & 12.32712354  & 14.83 \\ \hline
            \end{tabular}
        \caption{Table for Sample Salinity Test}
        \textit{Note: T denotes temperature, Calib Factor denotes Calibration Factor.}
        \label{table:salinity_measurements}
    \end{table}
\endgroup

From these measured values it can be seen that the probe has a measuring accuracy of approximately $\pm3.5$ \gls{psu}.
This inaccuracy could be attributed to noise and error.

\section{EIS and Machine Learning}
\subsection{Resistor-Capacitor Machine Learning Test}
As mentioned in Section~\ref{sec:ml_for_eis}, a Resistor-Capacitor circuit was modelled, to test the ability of the model in mapping impedance data to a given characteristic.
This model used the same input features, mapped to permittivity instead of salinity.
The dataset consisted of 300 data-points, with frequencies from $1-81Hz$ in increments of $20Hz$, permittivity $10-100F/m$, in increments of $10F/m$, and amplitudes $0.1-1.1V$ in increments of $0.2V$.
Noise was added to simulate real world conditions.
The random forest model works by creating multiple `decision trees' that learn patterns on the data, each tree makes a prediction, and the final prediction is the average of all the trees.
Derived features were additionally added, including, Angular Frequency $\omega$, $1/f$ and Capacitor Reactance $X_C$
Three random forest models were tested, with varying tree sizes, these being 30, 50 and 100. The results of these can be seen in table~\ref{table:rf_rc}, where it can be seen that the 100 tree model performed the best.

\begingroup
    \renewcommand{\arraystretch}{1.8} % increase row height (adjust factor as needed)
    \begin{table}[H]
        \centering
            \begin{tabular}{|>{\centering\arraybackslash}p{3cm}|
                >{\centering\arraybackslash}m{2cm}|
                >{\centering\arraybackslash}m{2cm}|
                >{\centering\arraybackslash}m{2cm}|}
            \hline
                \textbf{Model} & \textbf{$R^2$ Score} & \textbf{Mean Error (MAE)} & \textbf{$\%$ Error (MAPE)} \\ \hline
                100 trees & 0.9903 & 2.11  & 5.33$\%$ \\ \hline
                50 Trees  & 0.9898 & 2.21  & 5.06$\%$ \\ \hline
                30 Trees  & 0.9890 & 2.30  & 5.59$\%$ \\ \hline
            \end{tabular}
        \caption{Table for Random Forrest on RC Test Data}
        \label{table:rf_rc}
    \end{table}
\endgroup

\textbf{$R^2$ Score:} Measures how well the model predicts. 0.9903 mean it captures 99.03$\%$ of the variation in permittivity data, the high this number the better. \\
\textbf{Mean Absolute Error (MAE):} Average prediction error = 2.11  permittivity units. This means that predictions had an accuracy of $\pm2.11 F/m$. For a permittivity range of 0-100, this comes to 2.3$\%$. \\
\textbf{Mean Absolute Percentage Error (MAPE):} Average error in percentage. For example, if the predicted is 74.5 and the actual is 70, $\text{MAPE}=((|Actual-Predicted|)/Actual)\times100$.

The model identified which measurements were most important for predictions and found that impedance dominated with $73\%$ and inverse frequency $1/f$ played a role with $12\%$ whereas input voltage had barely any impact.
However this is due to RC circuit being a linear system.
The high importance of impedance and inverse frequency confirms that the model learned the physics, not just random correlations.

The model was then manually tested to verify the accuracy. These results can be seen in Table~\ref{table:rf_rc_test}, with a graph showing the predicted vs actual permittivity shown in Figure~\ref{fig:permit_graph}.

\begingroup
    \renewcommand{\arraystretch}{1.8} % increase row height (adjust factor as needed)
    \begin{table}[H]
        \centering
            \begin{tabular}{|>{\centering\arraybackslash}p{3cm}|
                >{\centering\arraybackslash}m{3cm}|
                >{\centering\arraybackslash}m{2cm}|}
            \hline
                \textbf{Actual $\epsilon_R$} & \textbf{Predicted $\epsilon_R$} & \textbf{Error} \\ \hline
                70 & 74.5 & 6.4$\%$ \\ \hline
                90 & 88.5 & 1.7$\%$ \\ \hline
                60 & 60.5 & 0.9$\%$ \\ \hline
                10 & 10.1 & 1.0$\%$ \\ \hline
                80 & 78.2 & 2.3$\%$ \\ \hline
            \end{tabular}
        \caption{Table for 100 Trees on manually inputted RC Data}
        \label{table:rf_rc_test}
    \end{table}
\endgroup

\begin{figure}[H]
    \centering
    \includegraphics[width=0.7\textwidth]{figures/permit_graph.png}
    \caption{Predicted vs Actual Salinity}
    \label{fig:permit_graph}
\end{figure}

This data showed that the model was feasible for salinity prediction.

\subsection{AC Wave Generation and Testing}
Before the probe could be used for \gls{eis}, test had to be conducted to ensure that it could reliably output a sine waveform, and to analyse how the signal was measured in water.
The sine wave generation was tested using the adc and an external oscilloscope. 
The oscilloscope was the Keysight Infiniivision DSOX2002A.
A 1Hz and 60Hz wave outputted by the \gls{dac} and measured, as seen in Figures~\ref{fig:oscilloscope} and~\ref{fig:adc_sine}.

\begin{figure}[H]
    \centering
    \begin{minipage}[t]{0.48\textwidth}
        \centering
        \includegraphics[width=\linewidth]{Figures/osc_1hz.jpg}
    \end{minipage}
    \hfill
    \begin{minipage}[t]{0.48\textwidth}
        \centering
        \includegraphics[width=\linewidth]{Figures/osc_60hz.jpg}
    \end{minipage}
    \caption{1Hz and 60Hz measured on the Oscilloscope}
    \label{fig:oscilloscope}
\end{figure}

\begin{figure}[H]
    \centering
    \begin{minipage}[t]{0.48\textwidth}
        \centering
        \includegraphics[width=\linewidth]{Figures/adc_1hz.png}
    \end{minipage}
    \hfill
    \begin{minipage}[t]{0.48\textwidth}
        \centering
        \includegraphics[width=\linewidth]{Figures/adc_60hz.png}
    \end{minipage}
    \caption{1Hz and 60Hz measured on the ADC}
    \label{fig:adc_sine}
\end{figure}

%typer here
A \gls{dc} offset can be observed, which is due to using the \gls{dac} to output the wave, causing it to only have positive voltages.

\subsection{Machine Learning Salinity Prediction Test}
To create the data for the prediction model, measurements were taken for frequencies $1-81Hz$, in increments of $20Hz$, voltages of $0.5-1.5V$ in increments of $0.2V$, and for 5 different salinity values.
It was intended for 10 salinity values to be captured, but the process proved longer than expected, limiting it to five samples.
This did prove to have an effect on the model as the smaller dataset meant less training data and a smaller training range.

Again three models were created, with 30, 50 and 100 trees. The results for these models are shown in Table~\ref{table:salinity_prediction_stats}.

\begingroup
    \renewcommand{\arraystretch}{1.8} % increase row height (adjust factor as needed)
    \begin{table}[H]
        \centering
            \begin{tabular}{|>{\centering\arraybackslash}p{3cm}|
                >{\centering\arraybackslash}m{2cm}|
                >{\centering\arraybackslash}m{2cm}|
                >{\centering\arraybackslash}m{2cm}|}
            \hline
                \textbf{Model} & \textbf{$R^2$ Score} & \textbf{Mean Error (MAE)} & \textbf{$\%$ Error (MAPE)} \\ \hline
                100 trees & -0.27 & 3.14  & 15.57$\%$ \\ \hline
                50 Trees  & -0.24 & 3.07  & 15.25$\%$ \\ \hline
                30 Trees  & -0.21 & 3.02  & 14.97$\%$ \\ \hline
            \end{tabular}
        \caption{Table for Random Forrest on Salinity Data}
        \label{table:salinity_prediction_stats}
    \end{table}
\endgroup

The negative $R^2$ score indicates that the model struggled to learn meaningful patterns from the data.
This can mainly be attributed to the low sample size and diversity.
However, this does not mean the model does not work. Individual predictions show some positive performance.
Despite the poor $R^2$ value, many individual predictions showed reasonable accuracy, as seen in Table~\ref{table:salinity_predictions}.

\begingroup
    \renewcommand{\arraystretch}{1.8} % increase row height (adjust factor as needed)
    \begin{table}[H]
        \centering
            \begin{tabular}{|>{\centering\arraybackslash}p{3cm}|
                >{\centering\arraybackslash}m{3cm}|
                >{\centering\arraybackslash}m{2cm}|
                >{\centering\arraybackslash}m{2cm}|}
            \hline
                \textbf{Actual $\epsilon_R$} & \textbf{Predicted $\epsilon_R$} & \textbf{Error} & \textbf{$\%$ Error} \\ \hline
                22.5 & 22.7 & 0.2 & 0.8$\%$ \\ \hline
                17.5 & 23.0 & 5.5 & 31.5$\%$ \\ \hline
                26.3 & 24.7 & 1.6 & 6.0$\%$ \\ \hline
            \end{tabular}
        \caption{Table for 100 Trees on Individual Salinity Predictions}
        \label{table:salinity_predictions}
    \end{table}
\endgroup

This accuracy, however, may be attributed to the low data variance.
Since most of the test data was close to the mean, the predictions remain near the mean.
This is illustrated in Figure~\ref{fig:salinity_graph}.

\begin{figure}[H]
    \centering
    \includegraphics[width=0.7\textwidth]{figures/salinity_graph.png}
    \caption{Predicted vs Actual Salinity}
    \label{fig:salinity_graph}
\end{figure}

The feature importance analysis does however show that the model correctly identified the impedance related features as the most important, demonstrating that it learned the physics, despite the limited overall performance.
The values for the feature importance analysis can be seen in Table~\ref{table:feature_analysis}.

\begingroup
    \renewcommand{\arraystretch}{1.8} % increase row height (adjust factor as needed)
    \begin{table}[H]
        \centering
            \begin{tabular}{|>{\centering\arraybackslash}p{4cm}|
                >{\centering\arraybackslash}m{2.8cm}|
                >{\centering\arraybackslash}m{4cm}|}
            \hline
                \textbf{Feature} & \textbf{Importance} & \textbf{Interpretation} \\ \hline
                $|H|$ (Transfer Magnitude) & 23.6$\%$ & Primary indicator of solution properties \\ \hline
                $H\angle$Transfer phase & 12.6$\%$ & Secondary impedance characteristic \\ \hline
                $1/(Z\times\omega)$ & 11.9$\%$ & Related to capacitance ($C \propto  1/(Z\times\omega)$) \\ \hline
                $|Z|$ (Impedance magnitude) & 11.1$\%$ & Direct measure of solution resistance \\ \hline
            \end{tabular}
        \caption{Table for Feature Analysis}
        \label{table:feature_analysis}
    \end{table}
\endgroup

From this it can be seen that despite the limited performance of the model, due to the low salinity variance and small dataset, the feature engineering correctly identified the physically meaningful features, and well represented salinities were predicted with good accuracy.
This proves that with sufficient data, the model should accurately predict salinity from \gls{eis} data.

\subsection{Additional AC Analysis}
While measuring impedance values for the salinity prediction dataset, it was noticed that between 1 and 20Hz there was a major decrease in the gain of the output signal.
This was investigated via a frequency response analysis.

In order to properly analyse the the data, a bode plot was created, as seen in Figure~\ref{fig:bode}

\begin{figure}[H]
    \centering
    \includegraphics[width=0.7\textwidth]{figures/bode_plot.png}
    \caption{Bode Plot}
    \label{fig:bode}
\end{figure}

The bode plot demonstrated tow key characteristics of the system. \\
\textbf{Magnitude Response:} As the frequency increased from 1-21Hz, the system gain progressively decreases, indicating a frequency dependant attenuation.
The salt water exhibits characteristic of a low pass filter, where higher frequency signals propagate less effectively than lower frequencies.
A significant reduction in magnitude is observed at 21Hz where the gain dips to -13dB, and remains around this level for higher frequencies. \\
\textbf{Phase Response:} The phase shift becomes increasingly negative as the frequency increases, showing that the output voltage lags the current.
The negative phase characteristic suggests capacitive behaviour becomes more dominant at higher frequencies.

This frequency response can be attributed to the electrical double layer formation between the electrode and the electrolyte, and the ionic polarisation within the saline solution.
The polarisation can also be attributed to the \gls{dac} being used to create the wave, since it only produces positive voltages, causing a semi-permanent charge on the elctrodes.
At low frequencies ions have sufficient time to respond to the applied electric field causing them to move according to their charge.
This results in efficient conduction and minimal phase lag.
As frequency increases, the electrode-solution interface acts similarly to a capacitor, due to charge accumulation on the electrodes.
At higher frequencies, the capacitive impedance becomes dominant, causing attenuation and phase lag.

%\include{Body/Discussion}
\chapter{Conclusions}
This project document the design and development of a conductivity based salinity measuring device, as well well investigating the feasibility of using \gls{eis} paired with \gls{ml} to predict salinity from impedance.
It showed the successful development of a probe that used electrodes to measure conductivity, and which used conductivity, coupled with temperature and pressure, through the PSS-78 equations, to measure salinity.
It also covered the development of the controller module, used to send instructions and receive readings from the probe board. These modules both performed successfully, showing accurate salinity measurements within $\pm3.5$ \gls{psu}, and transferring data and instructions accurately.
Two methods of salinity analysis were investigated, these being \gls{dc} analysus, through resistance measuring, and \gls{ac} analysis, through gls{eis} coupled with \gls{ml}.

The \gls{dc} analysis used two main measuring methods, a single voltage measurement, and a voltage sweep. 
These methods both return accurate resistance values, and showed good feasibility for salinity measurement.
However, further tests and investigations are required, to determine how this method can be improved.



\chapter{Recommendations}

Further circuit development for the probe is required.
Many of the components could be upgraded to more suitable ones, given the budget.
\gls{adc}s with a higher reference voltage will allow for a wider range of voltage measurements.
For the \gls{ac} signal generation, a method of removing the \gls{dc} component should be implemented.
This could be done using \gls{ac} coupling, including a virtual ground rail, or using a negative voltage.
Additionally, the implementation of a signal generator could solve this issue, as well as fill in where the \gls{dac} fell short in \gls{ac} signal generation, allowing for a wider range of frequencies at a higher resolution.

For the development of the machine learning algorithm to predict salinity, a much larger dataset with a higher variance should be used.
This would allow for a better range of training data, causing predictions to be less prone to error. The addition of temperature as a feature would also allow predictions to be done across a range of temperatures, however, this would require an even larger sample size.

The controller board served its purpose well, given the budget. However, further development should consider designing the board to work independent of a computer.
An interface with an OLED display, and built in menus and functions would allow for measurement without a computer.


%Use the IEEE numbered reference style for referencing your work as shown in your thesis guidelines.
%Please remember that the majority of your referenced work should be from journal articles, technical
%reports and books not online sources such as Wikipedia.


%Discuss the referencing style with your supervisor. When you are writing your document (if you are not using a citation editor) write the surname and date of the reference in square brackets when it is needed and highlight it as follows [Smith, 2004]. You can then return back to this later, update the numbers as they appear in text and remove the highlighting.
\cleardoublepage
\addcontentsline{toc}{chapter}{Bibliography}
\bibliographystyle{ieeetr}
\bibliography{References}

%\begin{thebibliography}{5}
%\bibitem{smt2011} M. S. Tsoeu and M. Braae, ``Control Systems,'' \emph{IEEE}, {\bf vol. 34(3)}, pp. 123-129, 2011.
%\bibitem{jct2010} J. C. Tapson, \emph{Instrumentation}, UCT Press, Cape Town, 2010.
%\end{thebibliography}

\appendix
\chapter{Proof of Graduate Attributes}

%\begin{center}
%\begin{tabular}{||p{2em} |p{15em} |p{20em}||}
% \hline
% \textbf{GA} & \textbf{Requirement} & \textbf{Justification and section in the report}  \\  [0.5ex] 
% \hline\hline
% 1 & Problem-solving & During this course I have done research on salinity measurements. This included techniques used for measuring salinity, their use cases and the mathematics behind conductivity methods used for salinity calculations. Using this research, I created a salinity measuring device that uses conductivity, temperature and depth (CTD) to measure salinity in salt water. This device required me to build a PCB and meet specific requirements, such as size and cost constraints. I had to carefully choose components and build the PCB with good practice methods used. I plan to then program this device to measure and calculate the salinity using the mathematics I researched. I have also researched Machine learning methods that can be applicable to creating a prediction or mapping model between the impedance of the salt water and its salinity. This then led to me researching Electrical Impedance Spectroscopy, which I found, should allow me to accurately create my ML model. \\  \hline
% 4 & Investigations, experiments and data analysis & I am designing a device that measures conductivity of saline solutions. Using this device, I will run an Electrical Impedance Spectroscopy measuring across the saline solution. Here I will compare the input wave to the output over the saline solution to calculate the impedance. This will be done over multiple different input waves and solutions of varying salinity to create a wide enough dataset to feed into the ML model. Here I will also use the probes to measure the salinity directly to find out the accuracy of the system and if any improvements should be made in future iterations.  \\
% \hline
% 5 & Use of engineering tools & For the hardware component of this project, I design a PCB. I used KiCAD to design both the schematic and the PCB. For the software component I plan to use VS Code and Arduino IDE with embedded C/C++. For Testing and debugging I will use tools including Oscilloscopes and Multimeters. For version control I have used Git and have a GitHub page for this project, allowing for easy changes and backing up of files. For machine learning I plan to use python with jupyter notebooks. \\
% \hline
% 6 & Professional and technical communication (Long report) & During my project I have been writing a report that documents the research I have done, the processes I have taken and my results. This report will be formatted according to the specified format and will be handed in at the end of the project. By documenting my project and meeting the deadlines I will show my ability for communication. My project also includes an oral presentation at the end which will show my presentation skills and verbal communication skills. Throughout this project I have and will submit all relevant tasks on time and have and plan to be punctual will all communication such as meetings. \\
% \hline
% 8 & Individual work & This project has shown that I have the ability to work individually, with research, design, experimentation and documentation. Where necessary I have attributed any work or ideas I have gotten from others to them. (i.e. I have referenced all sources) \\
% \hline
% 9 & Independent learning ability & have done significant research on salinity measurements, Electrical Impedance Spectroscopy, and Machine Learning algorithms. I have also designed a PCB, which involved learning about layered PCBs and interference, researching components, creating iterations to fix mistakes. This also included asking knowledgeable people such as my supervisor for advice on topics that I have some uncertainty on. \\ [1ex] 
% \hline
%\end{tabular}
%\end{center}

\begin{center}
\setlength{\extrarowheight}{1em} % optional: increases row height
\begin{longtable}{||p{2em}|p{15em}|p{20em}||}
\caption{Graduate Attributes and Justifications}\label{tab:GA_requirements} \\
\hline
\textbf{GA} & \textbf{Requirement} & \textbf{Justification and section in the report} \\ [0.5ex]
\hline
\endfirsthead

\hline
\textbf{GA} & \textbf{Requirement} & \textbf{Justification and section in the report} \\ [0.5ex]
\hline
\endhead

\hline
\multicolumn{3}{r}{\textit{Continued on next page}} \\
\hline
\endfoot

\hline
\endlastfoot

1 & Problem-solving &
During this course I have done research on salinity measurements. This included techniques used for measuring salinity, their use cases and the mathematics behind conductivity methods used for salinity calculations. Using this research, I created a salinity measuring device that uses conductivity, temperature and depth (CTD) to measure salinity in salt water. This device required me to build a PCB and meet specific requirements, such as size and cost constraints. I had to carefully choose components and build the PCB with good practice methods used. I plan to then program this device to measure and calculate the salinity using the mathematics I researched. I have also researched Machine learning methods that can be applicable to creating a prediction or mapping model between the impedance of the salt water and its salinity. This then led to me researching Electrical Impedance Spectroscopy, which I found, should allow me to accurately create my ML model. \\ \hline

4 & Investigations, experiments and data analysis &
I am designing a device that measures conductivity of saline solutions. Using this device, I will run an Electrical Impedance Spectroscopy measuring across the saline solution. Here I will compare the input wave to the output over the saline solution to calculate the impedance. This will be done over multiple different input waves and solutions of varying salinity to create a wide enough dataset to feed into the ML model. Here I will also use the probes to measure the salinity directly to find out the accuracy of the system and if any improvements should be made in future iterations. \\ \hline

5 & Use of engineering tools &
For the hardware component of this project, I design a PCB. I used KiCAD to design both the schematic and the PCB. For the software component I plan to use VS Code and Arduino IDE with embedded C/C++. For Testing and debugging I will use tools including Oscilloscopes and Multimeters. For version control I have used Git and have a GitHub page for this project, allowing for easy changes and backing up of files. For machine learning I plan to use python with jupyter notebooks. \\ \hline

6 & Professional and technical communication (Long report) &
During my project I have been writing a report that documents the research I have done, the processes I have taken and my results. This report will be formatted according to the specified format and will be handed in at the end of the project. By documenting my project and meeting the deadlines I will show my ability for communication. My project also includes an oral presentation at the end which will show my presentation skills and verbal communication skills. Throughout this project I have and will submit all relevant tasks on time and have and plan to be punctual will all communication such as meetings. \\ \hline

8 & Individual work &
This project has shown that I have the ability to work individually, with research, design, experimentation and documentation. Where necessary I have attributed any work or ideas I have gotten from others to them. (i.e. I have referenced all sources). \\ \hline

9 & Independent learning ability &
I have done significant research on salinity measurements, Electrical Impedance Spectroscopy, and Machine Learning algorithms. I have also designed a PCB, which involved learning about layered PCBs and interference, researching components, creating iterations to fix mistakes. This also included asking knowledgeable people such as my supervisor for advice on topics that I have some uncertainty on. \\ [1ex]
\hline

\end{longtable}
\end{center}

\chapter{Addenda}

\section{Examples of AI Usage}

\begin{figure}[H]
    \centering
    \includegraphics[width=0.6\textwidth]{figures/AI_Usage/Bib_file.png}
    \caption{Latex Referencing}
\end{figure}

\begin{figure}[H]
    \centering
    \includegraphics[width=0.6\textwidth]{figures/AI_Usage/latex_table.png}
    \caption{Latex Formatting}
\end{figure}

\begin{figure}[H]
    \centering
    \includegraphics[width=0.6\textwidth]{figures/AI_Usage/latex_figure.png}
    \caption{Latex Formatting}
\end{figure}

\begin{figure}[H]
    \centering
    \includegraphics[width=0.6\textwidth]{figures/AI_Usage/latex_pic.png}
    \caption{Latex Formatting}
\end{figure}

\begin{figure}[H]
    \centering
    \includegraphics[width=0.6\textwidth]{figures/AI_Usage/latex_gls.png}
    \caption{Latex Formatting}
\end{figure}

\begin{figure}[H]
    \centering
    \includegraphics[width=0.6\textwidth]{figures/AI_Usage/grammar1.png}
    \caption{Grammar}
    \textit{Note: Grammarly was also used for spelling and grammar checks}
\end{figure}

\begin{figure}[H]
    \centering
    \includegraphics[width=0.6\textwidth]{figures/AI_Usage/freezing_overview.png}
    \caption{Topic Overview}
\end{figure}

\begin{figure}[H]
    \centering
    \includegraphics[width=0.6\textwidth]{figures/AI_Usage/eis_overview.png}
    \caption{Topic Overview}
\end{figure}


\chapter{GitHub Repository}\label{app:c_github}
\href{https://github.com/zuhayrl/eee4022S_thesis}{Click here to access the GitHub repository.}


}} % for nohyphens and line spacing
\end{document}
