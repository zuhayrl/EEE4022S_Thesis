\chapter{Conclusions}
This project document the design and development of a conductivity based salinity measuring device, as well well investigating the feasibility of using \gls{eis} paired with \gls{ml} to predict salinity from impedance.
It showed the successful development of a probe that used electrodes to measure conductivity, and which used conductivity, coupled with temperature and pressure, through the PSS-78 equations, to measure salinity.
It also covered the development of the controller module, used to send instructions and receive readings from the probe board. These modules both performed successfully, showing accurate salinity measurements within $\pm3.5$ \gls{psu}, and transferring data and instructions accurately.
Two methods of salinity analysis were investigated, these being \gls{dc} analysus, through resistance measuring, and \gls{ac} analysis, through gls{eis} coupled with \gls{ml}.

The \gls{dc} analysis used two main measuring methods, a single voltage measurement, and a voltage sweep. 
These methods both return accurate resistance values, and showed good feasibility for salinity measurement.
However, further tests and investigations are required, to determine how this method can be improved.


