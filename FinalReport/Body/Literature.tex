\chapter{Literature Review}
\section{Introduction}
Accurate salinity measurement is fundamental to oceanographic research.
Traditional measurement techniques have evolved from labour-intensive chemical titration methods to modern electronic sensors, with electrical conductivity emerging as the predominant approach due to its combination of accuracy, speed, and practical deployability.
This literature review examines the current state of salinity measurement technology with particular emphasis on conductivity-based methods and emerging machine learning approaches for electrochemical data interpretation. The review is organised into three main sections.
First, we establish the fundamental concepts of salinity and provide a comprehensive comparison of available measurement techniques.
Second, we examine the theoretical foundations and practical implementation of electrical conductivity measurements for salinity determination, including instrumentation, calibration procedures, and current limitations.
Finally, we explore the application of \gls{eis} and \gls{ml} as advanced approaches for enhanced salinity analysis, examining how frequency-domain measurements and intelligent data processing can overcome limitations of traditional single-frequency conductivity methods.
This comprehensive review provides the theoretical and methodological foundation for developing a machine learning-enhanced impedance spectroscopy approach for salinity determination.

\section{Salinity: Definition}
Salinity is a fundamental characteristic of water, and is most commonly defined as the total amount of dissolved salts in water, and in the context of oceanography, seawater.
It is typically expressed in \gls{ppt} or \gls{psu}~\cite{salinity_def}. 
The concept of salinity has evoloved significantly from its early definition, which was based on chlorinity measurements.
Modern salinty is defined through the Practical Salinity Scale 1978 (PSS-78), where salinity is based on the conductivity ratio of standard seawater solutions, to a standard Potassium Chloride solution, and is dimensionless~\cite{unesco_salinity}.
The salinity-conductivity relationship is however, quite complex, requiring corrections and calibrations needed for depth and temperature, as these both play a factor in the conductivity of the water. 

\section{Overview of Salinity Measurement Methods}
There are a multitude of methods which can be used to measure salinity, each with their own advatages, limitations and levels of accuracy.
Traditional methods include gravimetric analysis, chemical titration (such as the Mohr-Knudsen method for chlorinity), and refractometry. While these techniques can provide accurate results, they are often time-consuming, require skilled operators, and are not easily adaptable to in-situ or automated measurements.
Modern approaches predominantly rely on electrical conductivity sensors, which offer rapid, repeatable, and automated salinity determination.
Other techniques, such as optical methods and ion-selective electrodes, have also been explored, but are less commonly used in oceanographic applications due to issues with robustness, calibration, or specificity.
The choice of method depends on the required accuracy, operational environment, and available resources.


\subsection{Historical Methods}
\subsubsection{Chlorinity Titration}
Early salinity measurements relied on chemical titration methods, in particular the Mohr-Knudsen chlorinity titration, which used silver nitrate.
The chlorinity of a solution has the definition `the mass of silver required to precipitate completely the halogens in $0.3 285 234 kg$ of sample seawater'.
This method was highly accurate, with results within ($\pm0.001$ PSU). However, it relied heavily on toxic chemicals, and was a time-consuming laboratory procedure, with limited practical application in the field~\cite{lewis_pss78}.

\subsubsection{Gravimetric Methods}
Gravimetric analysis, a technique used to determine an amount of a substance, by measuring its change in mass, invloves evaporation and the weighing of dissovled solids.
This method directly provided measurements of the salt content, within accuracies of ($\pm0.001$ PSU), under controlled laboratory conditions.
This method remains the reference standard for calibration processes, but is however, extremely slow~\cite{chemical_ocean}.

\subsection{Physical Property Based Methods}
There are several methods that utilise the relationship between salinity and the physical properties of water.

\subsubsection{Hydrometric and Density Methods}
Hydrometric methods using density measurements via hydrometers, offer salinity measurements that are low-cost, and electronics free. 
However, they are limited in precision with accuracies of $\pm 1-2$ PSU, and require large sample volumes.
The hydrometer is a floating instrument, that sinks to different depths depending on the density of the solution, and by measuring how high or low it floats, the density of the solution can be determined~\cite{the_globe_program_hydrometer}.
The following equation is used to map the relationship between salinity and density~\cite{kjerfve_density}.

\begin{equation}
    \rho = \rho_0(1+kS)
\end{equation}

where $\rho$ is the density, $\rho_0$ is the density of fresh water, $S$ is the salinity and $k$ is the proportionality constant. 

This can then be inverted to give Salinity from Density:

\begin{equation}
S = \frac{\frac{\rho}{\rho_0}-1}{k}
\end{equation}

This however, does not include temperature correction.

\subsubsection{Refractometric Thechniques}
Refractometric techniques measure the refractive index changes cuased by the dissovled salts.
The refractive index of seawater is influenced by wavelength, temperature, salinity, and pressure. 
Within the range of 500-700~nm wavelength, 0-30$^\circ$C temperature, 0-40~PSU salinity, and 0-11000~dbar pressure, the refractive index equation provides an accuracy of 0.4-80~ppm~PSU, with accuracy decreasing as pressure increases~\cite{refraction_millard}.
Refractometers, which require only a small sample volume, are compact devices, making them suitable for portable field measurements~\cite{malarde_high-resolution_refractometer}.
Fibre optic refractometers have improved portability and reduced temperature sensitivity, with moderate accuracy (±0.5-1 PSU), making them increasingly popular in aquaculture applications~\cite{zhang_high-temperature_fibre}. 


\subsubsection{Freezing Point Osmometry}
Freezing point depression osmometry exploits the colligative (i.e.~relating to the binding together of molecules) properties of dissolved salts.
The main principle relies on freezing point depression, which is the phenomenon where a solvents freeving point is lowered when a solute is added to it.
To perform the measurement, the water is cooled till its freezing point and the temperature drop is measured, which is then used to calculate the osmolality~\cite{freeze_abele}.
This method can achieve accuracies of $\pm2 mOsm/kg H_{2}O$ which is approximately $\pm0.1-0.2$ \gls{ppt}. However its requirement for precise temperature control limits its usage to laboratory applications~\cite{freezing_accuracy}.


\subsubsection{Magnetic Permeability}
Magnetic properties of liquids, particularly magnetic susceptibility, vary with ion concentration, offering a potential method for salinity determination.
Research has demonstrated that bulk magnetic susceptibility (BMS) of saline water correlates with salinity and conductivity measurements, with water quality parameters exhibiting an inverse relationship with magnetic susceptibility values.
This approach offers the benefit of non-contact measurement, potentially avoiding sample contamination or disturbance~\cite{rana_magnetic_2021}.
However, the technique requires sophisticated instrumentation, which are typically designed for laboratory use rather than field deployment. 



\subsection{Advanced Analytical Methods}
\subsubsection{Ion Chromatography}
Ion chromatography is an analytical technique used to separate and quantify ionic species in solution, making it highly valuable for determining the individual ion concentrations in seawater samples~\cite{ion_chromatography}.
The method works by passing a liquid sample through a column containing special resin beads that selectively hold onto different ions.
As a liquid solution flows through, ions are released at different times based on their properties and detected by measuring electrical conductivity.
For seawater analysis, ion chromatography can separately measure major ions like chloride, sulfate, sodium, magnesium, calcium, and potassium, providing detailed compositional data rather than just total salinity. 
The technique offers high precision and can detect ions at very low concentrations, though it requires more sophisticated equipment and longer analysis times compared to simpler methods~\cite{gros_ionic_2008}.

\section{Conductivity-Based Salinity Measurements}
\subsection{Theoretical Foundation}\label{sec:salinity_theory}
Electrical conductivity has emerged as the predominant method for salinity measurement due to its practical implementation, high accuracy and fast response time.
The technique utilises the strong correlation between dissolved ionic content and electrical conductivity.

The conductivity of a liquid is measured by its ability to conduct electrical current.
The relationship between conductivity and salinity is based on the concentration of dissolved ions in seawater.
The main ions found in sea water ($Na^+, Cl^-, Mg^{2+}, {SO_4}^{2-}, Ca^{2+}, K^+$) maintain a relatively constant proportional relationship, in ocean waters~\cite{chemical_ocean}.
This enables robust corrections between conductivity and total dissolved salt content.
Unlike other measurement techniques, conductivity accounts for all the ions in the water, not only chlorine, which is why it is considered a more accurate measure of salinity~\cite{salinity_def_calc}.

The Practical Salinity Scale 1978 (PSS-78) defines Practical Salinity $S_p$ through the conductivity ratio $K_{15}$, as shown below~\cite{teos-10}:

\begin{equation}\label{eqn:k15_salinity}
    K_{15} = \frac{C(S_p, 15, 0)}{C(KCl, 15, 0)}
\end{equation}

where the numerator, $C(S, 15, 0)$ represents the conductivity of seawater sample at 15$^\circ$C and standard atmospheric pressure ($1 atm/101.325 kPa/0dbar$), and the denomimator, $C(KCl, 15, 0)$ is the conductivity of a standard $KCl$ (Potassium Chloride) solution under identical temperature and pressure.
The standard $KCl$ solution consists of $32.4356 \times 10^{-3}kg$ of $KCl$ dissolved in $1kg$ of water~\cite{lewis_pss78}.
When the ratio between the water sample and the $KCl$ solution is 1, i.e. $K_{15} = 1$, then the Practical Salinity $S_p$ is, according to the definition, 35~\cite{teos-10}.

It is important to note that Practical Salinity is a unit-less quantity, and though it may be convenient, it would be incorrect to quote it in \gls{psu}. 
Practical salinity should rather be quoted as a certain Practical Salinity `on the Practical Salinity Scale PSS-78'~\cite{teos-10}. 

When $K_{15}$ does not equal 1, Practical Salinity, $S_p$ can be calculated using the equation below~\cite{teos-10}: 
\begin{equation}\label{eqn:salinity_short}
    S_p = \sum_{i=0}^{5}a_i{(K_{15})}^{i/2} 
\end{equation}

where $K_{15}$ is the equation defined above (Equation~\ref{eqn:k15_salinity}), and the coefficients $a_i$ are given in Table (\ref{tabel:pss-78}).


\subsection{Temperature and Pressure Compensation}\label{sec:salinity_compensation}
When calculating salinity at conditions other than 15$^\circ$C, and $0dbar$, the conductivity ratio $R$ is expanded to the product of three ratios $R_p$, $R_t$ and $r_t$ as follows~\cite{teos-10}:
\begin{equation}
    R=\frac{C(S_p, t, p)}{C(35, 15, 0)} = R_p R_t r_t
\end{equation}

where $t$, and $p$ are the temperature and pressure valid over the ranges $-2^{\circ}C \leq t \leq 35^{\circ}C$ and $0 \leq p \leq 10 000dbar$ respectively.

These ratios can be expanded as follows:

\begin{equation}
    R = \frac{C(S_p, t, p)}{C(35, 15^{\circ} C, 0)} = \frac{C(S_p, t, p)}{C(S_p, t, 0)} \cdot \frac{C(S_p, t, 0)}{C(35, t, 0)} \cdot \frac{C(35, t, 0)}{C(35, 15^{\circ} C, 0)} = R_p R_t r_t
\end{equation}

This equation represents the ratio between the conductivity measurement of a sample $C(S_p,t,p)$ and the conductivity of the standard solution $C(35, 15^{\circ}, 0)$~\cite{teos-10}. 
In order to find the salinity, $R_p$, $R_t$ and $r_t$ need to be caculated.
First, $r_t$ is calculated using the temperature of the sample:
\begin{equation}
    r_t = \sum_{i=0}^{4} {c_i}{t_i}
\end{equation}
$R_p$ is then calculated as a function of the temperature $t$, pressure $p$, and conductivity ratio $R$:
\begin{equation}
    R_p = 1 + \frac{\sum_{i=1}^{3}{e_i}{p^i}}{1+d_1{t}+d_2{t^2}+R[d_3+d_4 {t}]}
\end{equation}
Finally, $R_t$ can be evaluated using $R$, $R_p$ and $r_t$:
\begin{equation}
    R_t = \frac{R}{{R_p}{r_t}}
\end{equation} 

At standard conditions, i.e., temperature $t=15{^\circ}$C, $R_t$ is equal to $K_{15}$ an therefore Practical salinity $S_p$ can be calculated from Equation~\ref{eqn:k15_salinity}.
For cases where the temperature is not $t=15^\circ$C, Practical Salinity $S_p$ is given as a function of $R_t$, with $k=0.0162$~\cite{teos-10}:
\begin{equation}\label{eqn:salinity_full}
S_p = \sum_{i=0}^{5} a_i {(R_t)}^{i/2} + \frac{t-15}{1+k(t-15)} \sum_{i=0}^{5} b_i {(R_t)}^{i/2}
\end{equation}

Note that Equations~(\ref{eqn:k15_salinity}) to~(\ref{eqn:salinity_full}) are only valid in the range $2 < S_p < 42$, $-2^{\circ}C \leq t \leq 35^{\circ}C$ and $0 \leq p \leq 10 000dbar$.

\begingroup
    \renewcommand{\arraystretch}{1.8} % increase row height (adjust factor as needed)
    \begin{table}[h!]
        \centering
            \begin{tabular}{|>{\centering\arraybackslash}p{1cm}|
                >{\centering\arraybackslash}m{2cm}|
                >{\centering\arraybackslash}m{2cm}|
                >{\centering\arraybackslash}m{3cm}|
                >{\centering\arraybackslash}m{3cm}|
                >{\centering\arraybackslash}m{3cm}|}
            \hline
            $i$ & $a_i$ & $b_i$ & $c_i$ & $d_i$ & $e_i$ \\ \hline
            0 & $0.0080$ & $0.0005$ & $\num{6.766097e-1}$ &  &  \\ \hline
            1 & $-0.1692$ & $-0.0056$ & $\num{2.00564e-2}$ & $\num{3.426e-2}$ & $\num{2.070e-5}$ \\ \hline
            2 & $25.3851$ & $-0.0066$ & $\num{1.104259e-4}$ & $\num{4.464e-4}$ & $\num{-6.370e-10}$ \\ \hline
            3 & $14.0941$ & $-0.0375$ & $\num{-6.9698e-7}$ & $\num{4.215e-1}$ & $\num{3.989e-15}$ \\ \hline
            4 & $-7.0261$ & $0.0636$ & $\num{1.0031e-9}$ & $\num{-3.107e-3}$ &  \\ \hline
            5 & $2.7081$ & $-0.0144$ &  &  &  \\ \hline
            \end{tabular}
        \caption{Table of Coefficients for PSS-78 Equations~\cite{teos-10}}
        \label{tabel:pss-78}
    \end{table}
\endgroup

It must be noted that the PSS-78 equations use the IPTS-68 temperature scale and in order for them to work with the current ITS-90 scale, must be converted using the equation below~\cite{teos-10}:
\begin{equation}
    t_{68}^{\circ}C = 1.00024\times{t_{90}^{\circ}C}
\end{equation}


\subsection{Instrumentation and Technology}
The most common method for measuring salinity is by using a~\gls{ctd} device.
The fundamental concept of these devices involves placing two electrodes in a sample of water, applying a voltage across them and measuring the water's response. This is then paired with a temperature and depth correction, allowing for an accurate salinity measurement.
The depth value for these calculations is taken from the pressure at which the measurement is taken.
This pressure is then translated to depth using the standard depth to pressure equation~\cite{chip_based_ctd}.
Modern CTD systems achieve salinity accuracies better than $\pm0.005$ \gls{psu}, with some instruments like the Sea-Bird 911 Plus demonstrating historical accuracies of $\pm0.002$ \gls{psu} or $\pm0.0002$ \gls{psu}~\cite{chip_based_ctd}~\cite{ctd_accuracy}.

\subsection{Applications and Limitations}
Conductivity-based salinity measurements excel in most oceanographic and water quality applications due to its accuracy, speed, and reliability.
The conductivity method allows for real-time data capture, continuous monitoring, and easy integration with autonomous devices~\cite{roemmich_argo_2009}. 

However, this method does face some limitations. Due to its dependance on the water's capacity to conduct electricity, freshawater applications require specialised low-conductivity sensors, while hyper-saline environments could exceed the standard calibration range.
The method's reliance on emperical correlations derived from typical seawater compositions can introduce errors in waters ocean waters affected by external factors such as pollution or freshwater inflow from connecting rivers, which can alter ionic composition and introduce variability not captured by standard seawater-based calibrations.
In such environments supplementry practices may be necessary for accurate salinity measurements~\cite{uncles_estuarine_2002}.

\section{Machine Learning Applications in Electrochemical \\ Impedance Spectroscopy}
\subsection{\gls{eis}}

\glsfirst{eis} is an analytical technique used to characterise the electrical properties of materials and interfaces, usually electrode or an electrolyte, by measuring impedance (opposition to current flow) across a range of frequencies~\cite{canales_electrochemical_2021}.
In essence, EIS describes the electrode behavior in the presence of an electrolyte in terms of electrical parameters, including resistances and capacitances.
It involves measuring the system's response to an applied electrical signal, and then transforming these time-domain signals into the frequency domain.
The technique is based on applying an /gls{ac} signal to the electrodes and determining the corresponding response.
Unlike simple conductivity measurements that capture only resistive properties at a single frequency, EIS measures both the magnitude and phase of impedance across a range of frequencies, typically from 0.001 Hz to 1 MHz~\cite{barsoukov_impedance_2005}.

The complex impedance $Z(\omega)=Z'(\omega)+jZ''(\omega)$ contains both resistive (real) and reactive (imaginary) components, that respond differently to ionic concentration, species mobility, and electrode interface effects.
High-frequency impedance primarily reflects bulk solution resistance, while lower frequencies reveal interfacial (between two faces) phenomena including double-layer capacitance and charge transfer resistance~\cite{orazem_electrochemical_2008}.

EIS has demonstrated success in quantitative concentration analysis across diverse solution systems.
In electrolyte analysis, multi-frequency measurements enable determination of ionic strength through characteristic impedance signatures that are less sensitive to temperature variations and electrode fouling than DC conductivity measurements~\cite{barsoukov_impedance_2005}.


\subsection{EIS Fundamentals}
The \gls{eis} measuring process invloves applying an AC signal at a specific frequency and amplitude to a sample.
By sweeping across a multitude of frequencies, a complete impedance spectrum can be obtained.
The impedance magnitude $|Z|$ and phase angle $\phi$ at each frequency is recorded, and are used to analyse the electrical properties of the solution~\cite{orazem_electrochemical_2008}.
For salinity, it can be used to characterise the electrical properties of ionic solutions, including seawater, by analyzing how ions affect the impedance spectrum. 
This data can be used to approximate the system using \glspl{eec}, which are circuits ussed to describe a system with discrete electrical circuit elements~\cite{barsoukov_impedance_2005}.
\glspl{eec} fulfill the overall goal of \gls{eis} is to describe the behaviour of the electrode and solution in terms of electrical parameters such as resistances and capacitances~\cite{canales_electrochemical_2021}.

%\section{Machine Learning for Impedance to Salinity Mapping}
\subsection{Limitations of Traditional Equivalent Circuit Modelling}
Traditional impedance analysis relies on the modelling of \glspl{eec}, to derive the salinity from the impedance.
This process requires extensive electrochemical expertise, in order to understand the chemical composition of the solution.
Manual parameter extraction from impedance spectra is time consuming and may fail to capture subtle features that correlate with salinity.
Linear calibration methods assume simple relationships between impedance measurements and concentration, which may not hold across wide salinity ranges or in solutions with varying ionic compositions.
Temperature effects and electrode aging can further complicate traditional calibration approaches~\cite{barsoukov_impedance_2005}.

\subsection{Machine Learning Modelling for Impedance Analysis}
Machine learning approaches offer advantages by learning complex non-linear relationships directly from experimental data without the need for explicit physical models or \glspl{eec}.
Supervised learning algorithms can map measured impedance values at specific input signal frequencies and amplitudes to target salinity concentrations through automated pattern recognition.
The general framework involves training a model on a dataset of known salinity samples, where each sample is characterised by its impedance response to AC excitation at defined frequencies and amplitudes~\cite{bongiorno_exploring_2022}. 
The trained model can then predict salinity from new impedance measurements, effectively learning the impedance-salinity mapping function.
Machine learning eliminates subjective bias inherent in manual impedance interpretation and consistently applies identical analysis procedures across all measurements.
The algorithms can identify complex patterns in multi-frequency impedance data that may be overlooked in traditional analysis~\cite{xu_electrochemical_2020}.

\subsection{Machine Learning Algorithms for Salinity Prediction}
\subsubsection{Neural Networks}
Neural networks are machine learning programs that make decisions similarly to the human brain, using processes that mimic how biological neurons work together to identify patterns, weigh options, and reach conclusions~\cite{ibm_nn}.

Every neural network consists of layers of nodes (artificial neurons), these being an input layer, one or more hidden layers, and an output layer. Each node connects to others and has its own weight and threshold value.
When a node's output exceeds its threshold value, it activates and sends data to the next layer, if not, no data passes through.
This creates a feed-forward process where information flows from input to output. Each node functions like a linear regression model with inputs, weights, a bias, and an output.
Weights determine how important each input is, with larger weights having greater significance. Inputs are multiplied by their weights, summed together, and passed through an activation function to determine if the node fires.
Neural networks rely on training data to learn and improve accuracy over time.
They use cost functions to evaluate accuracy and adjust their weights and biases through gradient descent and backpropagation to minimize errors and reach optimal performance~\cite{ibm_nn}.

\glspl{ann} represent the most widely adopted machine learning approach for impedance-based concentration prediction due to their exceptional capability for modelling complex non-linear relationships.
The architecture typically consists of an input layer receiving impedance features such as frequency and phase, one or more hidden layers that extract relevant patterns, and an output layer producing predictions~\cite{christopher_m_bishop_pattern_2006}.
Hidden layers with non-linear activation functions enable the network to capture complex impedance-salinity relationships~\cite{ann_eis_hsueh}.

Deep neural networks with multiple hidden layers can learn hierarchical representations, potentially identifying frequency dependent patterns at different levels.
However, they require larger training datasets and careful regularisation to prevent overfitting~\cite{ann_eis_hsueh}.

\subsubsection{Random Forrest Algorithms}
Random forest is a machine learning algorithm which combines the output of multiple decision trees to reach a single result. It handles both classification and regression problems.
The model is built from multiple decision trees. Decision trees start with a basic question and use a series of questions (decision nodes) to split data, ultimately leading to a final decision at the leaf node. They're trained through the Classification and Regression Tree (CART) algorithm.
Unlike single decision trees that can be prone to bias and overfitting, when multiple uncorrelated decision trees form an ensemble in random forest, they predict more accurate results.
By accounting for potential variability in the data through feature randomness, random forests reduce the risk of overfitting, bias, and overall variance, resulting in more precise predictions~\cite{ibm_randomforrest}.

Random Forest methods demonstrate excellent performance with mixed data types, enabling simultaneous utilisation of raw impedance values, derived electrochemical parameters, and statistical features within unified prediction models.
The algorithm's resistance to overfitting makes it particularly suitable for limited training data scenarios common in analytical applications~\cite{salman_random_2024}.

