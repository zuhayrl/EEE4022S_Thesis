\chapter{Literature Review}
\section{Introduction}
Accurate salinity measurement is fundamental to oceanographic research.
Traditional measurement techniques have evolved from labour-intensive chemical titration methods to modern electronic sensors, with electrical conductivity emerging as the predominant approach due to its combination of accuracy, speed, and practical deployability.
This literature review examines the current state of salinity measurement technology with particular emphasis on conductivity-based methods and emerging machine learning approaches for electrochemical data interpretation. The review is organised into three main sections.
First, we establish the fundamental concepts of salinity and provide a comprehensive comparison of available measurement techniques.
Second, we examine the theoretical foundations and practical implementation of electrical conductivity measurements for salinity determination, including instrumentation, calibration procedures, and current limitations.
Finally, we explore the application of \gls{eis} and \gls{ml} as advanced approaches for enhanced salinity analysis, examining how frequency-domain measurements and intelligent data processing can overcome limitations of traditional single-frequency conductivity methods.
This comprehensive review provides the theoretical and methodological foundation for developing a machine learning-enhanced impedance spectroscopy approach for salinity determination.

\section{Salinity: Definition}
Salinity is a fundamental characteristic of water, and is most commonly defined as the total amount of dissolved salts in water, and in the context of oceanography, seawater.
It is typically expressed in \gls{ppt} or \gls{psu}~\cite{salinity_def}. 
The concept of salinity has evoloved significantly from its early definition, which was based on chlorinity measurements.
Modern salinty is defined through the Practical Salinity Scale 1978 (PSS-78), where salinity is based on the conductivity ratio of standard seawater solutions, to a standard Potassium Chloride solution, and is dimensionless~\cite{unesco_salinity}.
The salinity-conductivity relationship is however, quite complex, requiring corrections and calibrations needed for depth and temperature, as these both play a factor in the conductivity of the water. 

\section{Overview of Salinity Measurement Methods}
There are a multitude of methods which can be used to measure salinity, each with their own advatages, limitations and levels of accuracy.
Traditional methods include gravimetric analysis, chemical titration (such as the Mohr-Knudsen method for chlorinity), and refractometry. While these techniques can provide accurate results, they are often time-consuming, require skilled operators, and are not easily adaptable to in-situ or automated measurements.
Modern approaches predominantly rely on electrical conductivity sensors, which offer rapid, repeatable, and automated salinity determination.
Other techniques, such as optical methods and ion-selective electrodes, have also been explored, but are less commonly used in oceanographic applications due to issues with robustness, calibration, or specificity.
The choice of method depends on the required accuracy, operational environment, and available resources.


\subsection{Historical Methods}
\subsubsection{Chlorinity Titration}
Early salinity measurements relied on chemical titration methods, in particular the Mohr-Knudsen chlorinity titration, which used silver nitrate.
The chlorinity of a solution has the definition `the mass of silver required to precipitate completely the halogens in $0.3 285 234 kg$ of the ocean-water sample'
This method was highly accurate, with results within ($\pm0.001$ PSU). However, it relied heavily on toxic chemicals, and was a time-consuming laboratory procedure, with limited practical application in the field.

\subsubsection{Gravimetric Methods}
Gravimetric analysis, a technique used to determine an amount of a substance, by measuring its change in mass, invloves evaporation and the weighing of dissovled solids.
This method directly provided measurements of the salt content, within accuracies of ($\pm0.001$ PSU), under controlled laboratory conditions.
This method remains the reference standard for calibration processes, but is however, extremely slow.

\subsection{Physical Property Based Methods}
There are several methods that utilise the relationship between salinity and the physical properties of water.

\subsubsection{Hydrometric and Density Methods}
Hydrometric methods using density measurements via hydrometers, offer salinity measurements that are low-cost, and electronics free. 
However, they are limited in precision with accuracies of $\pm 1-2$ PSU, and require large sample volumes.
The hydrometer is a floating instrument, that sinks to different depths depending on the density of the solution, and by measuring how high or low it floats, the density of the solution can be determined.
The following equation is used to map the relationship between salinity and density.

\begin{equation}
    \rho = \rho_0(1+kS)
\end{equation}

where $\rho$ is the density, $\rho_0$ is the density of fresh water, $S$ is the salinity and $k$ is the proportionality constant. 

This can then be inverted to give Salinity from Density:

\begin{equation}
S = \frac{\frac{\rho}{\rho_0}-1}{k}
\end{equation}

This however, does not include temperature correction.

\subsubsection{Refractometric Thechniques}
Refractometric techniques measure the refractive index changes cuased by the dissovled salts.
The refractive index of seawater is influenced by wavelength, temperature, salinity, and pressure. 
Within the range of 500-700~nm wavelength, 0-30$^\circ$C temperature, 0-40~PSU salinity, and 0-11000~dbar pressure, the refractive index equation provides an accuracy of 0.4-80~ppm~PSU, with accuracy decreasing as pressure increases.
Refractometers, which require only a small sample volume, are compact devices, making them suitable for portable field measurements.
Fibre optic refractometers have improved portability and reduced temperature sensitivity, with moderate accuracy (±0.5-1 PSU), making them increasingly popular in aquaculture applications. 


\subsubsection{Freezing Point Osmometry}
Freezing point depression osmometry exploits the colligative (i.e.~relating to the binding together of molecules) properties of dissolved salts.
The main principle relies on freezing point depression, which is the phenomenon where a solvents freeving point is lowered when a solute is added to it.
To perform the measurement, the water is cooled till its freezing point and the temperature drop is measured, which is then used to calculate the osmolality.
This method can achieve accuracies as high as ($\pm0.001$ PSU), however its requirement for precise temperature control limits its usage to laboratory applications.

\subsubsection{Interferometry}
Interferometry is a measurement technique which measures how electro-magnetic waves are affected by changes in a solution.
Two identical light waves are passed through two solutions, one benchmark and one test sample solution.
The gain and phase shift between the waves is then used to calculate the salinity. 
This method requires precisely aligned mirrors to direct the light waves, causing it to be relatively large.

\subsubsection{Electromagnetic Induction and Magnetic Permeability}


\subsection{Advanced Analytical Methods}


\subsection{Remote Autonomous Sensing}


\section{Conductivity-Based Salinity Measurements}
\subsection{Theoretical Foundation}
Electrical conductivity has emerged as the predominant method for salinity measurement due to its practical implementation, high accuracy and fast response time.
The technique utilises the strong correlation between dissolved ionic content and electrical conductivity.

The conductivity of a liquid is measured by its ability to conduct electrical current.
The relationship between conductivity and salinity is based on the concentration of dissolved ions in seawater.
The main ions found in sea water ($Na, Cl, Mg^2, SO^2, Ca^2 and K$) maintain a relatively constant proportional relationship, in ocean waters.
This enables robust corrections between conductivity and total dissolved salt content.
Unlike other measurement techniques, conductivity accounts for all the ions in the water, not only chlorine, which is why it is considered a more accurate measure of salinity.
The Practical Salinity Scale 1978 (PSS-78) defines salinity through the conductivity ratio, as shown below:

\begin{equation}
    K_{15} = \frac{C(S, 15, 0)}{C(KCl, 15, 0)}
\end{equation}

where $C(S, 15, 0)$ represents the conductivity of sewater sample $S$ at 15$^\circ$C and standard atmospheric pressure (1 atm/101.325 kPa), and $C(KCl, 15, 0)$ is the conductivity of a standard $KCl$ solution under identical conditions.

\subsection{Temperature Compensation}

