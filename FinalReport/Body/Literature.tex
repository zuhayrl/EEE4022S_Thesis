\chapter{Literature Review}
\section{Introduction}
Accurate salinity measurement is fundamental to oceanographic research.
Traditional measurement techniques have evolved from labour-intensive chemical titration methods to modern electronic sensors, with electrical conductivity emerging as the predominant approach due to its combination of accuracy, speed, and practical deployability.
This literature review examines the current state of salinity measurement technology with particular emphasis on conductivity-based methods and emerging machine learning approaches for electrochemical data interpretation. The review is organised into three main sections.
First, we establish the fundamental concepts of salinity and provide a comprehensive comparison of available measurement techniques.
Second, we examine the theoretical foundations and practical implementation of electrical conductivity measurements for salinity determination, including instrumentation, calibration procedures, and current limitations.
Finally, we explore the application of \gls{eis} and \gls{ml} as advanced approaches for enhanced salinity analysis, examining how frequency-domain measurements and intelligent data processing can overcome limitations of traditional single-frequency conductivity methods.
This comprehensive review provides the theoretical and methodological foundation for developing a machine learning-enhanced impedance spectroscopy approach for salinity determination.

\section{Salinity: Definition}
Salinity is a fundamental characteristic of water, and is most commonly defined as the total amount of dissolved salts in water, and in the context of oceanography, seawater.
It is typically expressed in \gls{ppt} or \gls{psu}~\cite{salinity_def}. 
The concept of salinity has evoloved significantly from its early definition, which was based on chlorinity measurements.
Modern salinty is defined through the Practical Salinity Scale 1978 (PSS-78), where salinity is based on the conductivity ratio of standard seawater solutions, to a standard Potassium Chloride solution, and is dimensionless~\cite{unesco_salinity}.
The salinity-conductivity relationship is however, quite complex, requiring corrections and calibrations needed for depth and temperature, as these both play a factor in the conductivity of the water. 

\section{Overview of Salinity Measurement Methods}
There are a multitude of methods which can be used to measure salinity, each with their own advatages, limitations and levels of accuracy.
Traditional methods include gravimetric analysis, chemical titration (such as the Mohr-Knudsen method for chlorinity), and refractometry. While these techniques can provide accurate results, they are often time-consuming, require skilled operators, and are not easily adaptable to in-situ or automated measurements.
Modern approaches predominantly rely on electrical conductivity sensors, which offer rapid, repeatable, and automated salinity determination.
Other techniques, such as optical methods and ion-selective electrodes, have also been explored, but are less commonly used in oceanographic applications due to issues with robustness, calibration, or specificity.
The choice of method depends on the required accuracy, operational environment, and available resources.


\subsection{Historical Methods}
Early salinity measurements relied on chemical titration methods, in particular the Mohr-Knudsen chlorinity titration, which used silver nitrate.
This method was highly accurate, with results within (\pm0.001 PSU). However, this method relied heavily on toxic chemicals, and was a time-consuming laboratory procedure, with limited practical application in the field.

