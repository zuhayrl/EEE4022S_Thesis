\chapter{Recommendations}

Further circuit development for the probe is required.
Many of the components could be upgraded to more suitable ones, given the budget.
\gls{adc}s with a higher reference voltage will allow for a wider range of voltage measurements.
For the \gls{ac} signal generation, a method of removing the \gls{dc} component should be implemented.
This could be done using \gls{ac} coupling, including a virtual ground rail, or using a negative voltage.
Additionally, the implementation of a signal generator could solve this issue, as well as fill in where the \gls{dac} fell short in \gls{ac} signal generation, allowing for a wider range of frequencies at a higher resolution.

For the development of the machine learning algorithm to predict salinity, a much larger dataset with a higher variance should be used.
This would allow for a better range of training data, causing predictions to be less prone to error. The addition of temperature as a feature would also allow predictions to be done across a range of temperatures, however, this would require an even larger sample size.

The controller board served its purpose well, given the budget. However, further development should consider designing the board to work independent of a computer.
An interface with an OLED display, and built in menus and functions would allow for measurement without a computer.

