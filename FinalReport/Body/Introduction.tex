\chapter{Introduction}

\section{Background to the study}
Antarctic ice shelves form where glaciers from the continent extend out over the ocean.
This creates massive floating platforms that serve as crucial boundaries between land, ice and the marine environment.
At the base of these ice shelves, the relatively warmer ocean water melts the ice above, releasing freshwater that rises upward because it is less dense than the surrounding salty seawater.
As this freshwater moves upwards and experiences less pressure, the pressure difference can allow it to become chilled below its normal freezing point without turning to ice.
When this water eventually refreezes onto the underside of the ice shelf, it creates what is known as marine ice.
Marine ice is a layer of ice that originated from seawater, whereas regular ice forms from snow that fell on land.
During this refreezing process, salt is pushed out of the water, forming ice crystals in a phenomenon known as brine rejection, which creates pockets of extremely salty, dense water that sink downwards.
This phenomenon helps drive ocean currents that continually bring new water into contact with the ice shelf base, establishing a continuous cycle of melting and freezing.
Understanding how salinity varies within and beneath ice shelves is increasingly important because changes in ocean temperature and salt content can significantly affect how quickly the ice melts from below.
This melting process can weaken ice shelves and, in extreme cases, lead to their collapse.
This, in turn, allows glaciers on land to flow more rapidly into the ocean, contributing to sea level rise.
Traditionally, scientists have measured salinity in these environments by collecting water samples at various depths.
However, this approach only provides salinity data at specific locations and times and does not allow for continuous measurement.

\section{Objectives of this study}
This study aims to design a prototype device that would allow researchers to take continuous real time salinity measurements in these `water towers'.
It aims to create a device that can accurately measure salinity, and that can be iterated upon, allowing it to be used in the harsh conditions of the Antarctic.
Additionally, this study aims to create a machine learning model, which would allow for the prediction of salinity from the electro-chemical make-up of the water, through a process called Electrochemical Impedance Spectroscopy.

\section{Scope and Limitations}
The scope of this project includes the design, assembly and testing of the salinity measuring device, as well as a review on the feasibility of using machine learning to predict salinity.
This includes researching relevant literature, to get a good understanding of similar devices that already exist and how salinity can be calculated using these devices, followed by the design process and assembly of the prototype device, and the design of the machine learning model.
This then moves to the testing of the device and the machine learning model, to determine their effectiveness in measuring and predicting salinity, respectively.

This project must be completed in a specified time of 13 weeks, from its inception to submission.
A budget of R2000 has been imposed on the entire project. This includes design, assembly and testing.
This budget can only be spent through the Electrical Engineering Department of the University of Cape Town. 


\section{Plan of development}
This projects first starts with an literature review in Chapter 2, where salinity, its measurement methods, what \gls{eis} is and how it can be paired with \gls{ml}, are reviewed.
Chapter 3 covers the chosen methodology, covering the choice of salinity measurement method, its design and assembly.
Chapter 4 details the testing and evaluation of the chosen design, including its salinity measuring accuracy.
Chapter 5 concludes the report with a summary of the objectives and results, and Chapter 6 includes recommendations for further studies on this work.
